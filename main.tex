% !TEX encoding = IsoLatin

% La riga soprastante serve per configurare gli editor TeXShop, TeXWorks
% e TeXstudio per gestire questo file con la codifica IsoLatin o Latin 1
% o ISO 8859-1.

% per commentare una riga mettere % al suo inizio
% per s-commentare una riga (ossia attivarla) togliere il % al suo inizio
%
\documentclass[pdfa% formato PDF/A, obbligatorio per l'archiviazione delle tesi di Polito
%,cucitura%lascia margine per la rilegatura
,twoside% per stampa fronte-retro (fortemente consigliato per tesi voluminose, opzionale per le altre)
,12pt% font pi? grande (12pt) rispetto a quello normalmente usato (11pt)
]{toptesi}
%
\PassOptionsToPackage{dvipsnames}{xcolor}
\usepackage{url}
\usepackage{breakurl}
\usepackage[breaklinks]{hyperref}
\def\UrlBreaks{\do\/\do-}
\hypersetup{%
    pdfpagemode={UseOutlines},
    bookmarksopen,
    pdfstartview={FitH},
    colorlinks,
    linkcolor={blue},
    citecolor={red},
    urlcolor={blue}
  }
% \documentclass[11pt,twoside,oldstyle,autoretitolo,classica,greek]{toptesi}
% \usepackage[or]{teubner}
%%%%%%%%%%%%%%%%%%%%%%%%%%%%%%%%%%%%%%%%%%%%%%%%%%%%
%
% Esempio di composizione di tesi di laurea.
%
% Questo esempio e' stato preparato inizialmente 13-marzo-1989
% e poi e' stato modificato via via che TOPtesi andava
% arricchendosi di altre possibilita'.
%
% Nel seguito laurea "quinquennale" sta anche per "specialistica" o "magistrale"

% Cambiare encoding a piacere; oppure non caricare nessun encoding se si usano
% solo caratteri a 7 bit (ASCII) nei file d'entrata.
%
\usepackage[latin1]{inputenc}% IMPORTANTE! usare codifica ISO-8859-1 per le lettere accentate
% \usepackage[outputdir=out,cache=false]{minted}

\input{other/commands.tex}
\usepackage{cryptocode}
\usepackage{svg}
\usepackage{plantuml}
% modificare il path di inkscape nel caso di errori
\setsvg{inkscape={"C:/Program Files/Inkscape/bin/inkscape.com"}}
\svgpath{{chapters/images}}

\usepackage{enumitem}
\usepackage{booktabs}
\usepackage{fontawesome}
\usepackage{multicol}
\usepackage{xcolor}
\usepackage{tcolorbox}
\newtcolorbox{mybox}{colback=red!5!white,colframe=red!75!black}
\lstdefinestyle{json}{
  string=[s]{"}{"},
  stringstyle=\color{blue},
  comment=[l]{:},
  commentstyle=\color{black},
}
\lstdefinestyle{DOS}{
  backgroundcolor=\color{black},
  basicstyle=\color{white}\ttfamily
}

\makeatletter
\def\lst@visiblespace{\lst@ttfamily{\phantom{\char32}}{\phantom{\textvisiblespace}}}
\begin{document}
\selectlanguage{english}


\ateneo{Politecnico di Torino}

%%% scegliere la propria facolt? (solo PRIMA dell'AA 2012-2013)
%
%\facolta[III]{Ingegneria dell'Informazione}
%\facolta[IV]{Organizzazione d'Impresa\\e Ingegneria Gestionale}
%\Materia{Remote sensing}% uso sconsigliato

%\monografia{Gestione informatizzata di un magazzino ricambi}% per la laurea triennale
\titolo{Self-Sovereign-Identity as a Service}% per la laurea quinquennale e il dottorato
\sottotitolo{Trusted computation offloading for IoT constrained devices}% NON obbligatorio, per la laurea quinquennale e il dottorato

%%% scegliere il proprio corso
%
%\corsodilaurea{Ingegneria dell'Organizzazione d'Impresa}% per la laurea di primo e secondo livello
%\corsodilaurea{Ingegneria Logistica e della Produzione}% per la laurea di primo e secondo livello
%\corsodilaurea{Ingegneria Gestionale}% per la laurea di primo e secondo livello
\corsodilaurea{Computer Engineering}% per la laurea di primo e secondo livello
%\corsodidottorato{Meccanica}% per il dottorato

\candidato{Luca \textsc{Giorgino}}% per tutti i percorsi
%\secondocandidato{Evangelista \textsc{Torricelli}}% per la laurea magistrale solamente
%\direttore{prof. Albert Einstein}% per il dottorato
%\coordinatore{prof. Albert Einstein}% per il dottorato
\relatore{prof.\ Antonio Lioy}% per la laurea e il dottorato
%\secondorelatore{dipl.~ing.~Werner von Braun}% per la laurea magistrale
%\terzorelatore{{\tabular{@{}l}dott.\ Neil Armstrong\\prof. Maria Rossi\endtabular}}% per la laurea magistrale
%\tutore{ing.~Karl Von Braun}% per il dottorato
%\tutoreaziendale{PhD.\ ing.\ Andrea Vesco} % solo per la laurea di secondo livello con tesi svolta in azienda
\tutoreaziendale{Andrea Vesco, Ph.D.\\Alberto Carelli, Ph.D.} % solo per la laurea di secondo livello con tesi svolta in azienda
% \NomeTutoreAziendale{Supervisore aziendale\\LINKS FOUNDATION}
%\sedutadilaurea{Ottobre 2022}% per la laurea quinquennale
%\esamedidottorato{Novembre 1610}% per il dottorato
%\sedutadilaurea{\textsc{Ottobre} 2022}% per la laurea triennale
\sedutadilaurea{\textsc{Academic~year} 2021-2022}% per la laurea magistrale
%\annoaccademico{1615-1616}% solo con l'opzione classica
%\annoaccademico{2006-2007}% idem
%\ciclodidottorato{XV}% per il dottorato
\logosede{other/logopolito}
%
%\chapterbib %solo per vedere che cosa succede; e' preferibile comporre una sola bibliografia
%\AdvisorName{Supervisors}
%\newtheorem{osservazione}{Osservazione}% Standard LaTeX

%\usepackage[a-1b]{pdfx}
%\hypersetup{%
%    pdfpagemode={UseOutlines},
%    bookmarksopen,
%    pdfstartview={FitH},
%    colorlinks,
%    linkcolor={blue},
%    citecolor={green},
%    urlcolor={blue}
%  }

%
% per numerare e far comparire nell'indice anche le sezioni di quarto livello
% SCONSIGLIATO! da usarsi solo in caso di estrema necessit?
%\setcounter{secnumdepth}{4}% section-numbering-depth
%\setcounter{tocdepth}{4}% TOC-numbering-depth (TOC=Table-Of-Content)

%\setbindingcorrection{3mm}

\errorcontextlines=9

\iflanguage{english}{
  %\retrofrontespizio{This work is subject to the Creative Commons Licence}
  \DottoratoIn{PhD Course in\space}
  \CorsoDiLaureaIn{Master degree course in\space}
  \NomeMonografia{Bachelor Degree Final Work}
  \TesiDiLaurea{Master Degree Thesis}
  \NomeDissertazione{PhD Dissertation}
  \InName{in}
  \CandidateName{Candidate}
  \AdvisorName{Supervisor}
  \TutorName{Tutor}
  \NomeTutoreAziendale{Tutors\\LINKS FOUNDATION}
  \CycleName{cycle}
  \NomePrimoTomo{First volume}
  \NomeSecondoTomo{Second Volume}
  \NomeTerzoTomo{Third Volume}
  \NomeQuartoTomo{Fourth Volume}
}{}

\frontespizio
\paginavuota
\newpage
%per sfruttare meglio lo spazio nella pagina
\advance\voffset -5mm
\advance\textheight 30mm

% opzionale, solo se si vuole dedicare la tesi a delle persone care
% \begin{dedica}
% To my parents
% \end{dedica}

\sommario

% Digital identity makes IoT objects unique and distinguishes them from each other. Self-Sovereign Identity aims to provide a digital identity that is both verified and verifiable to IoT objects while building a digital ecosystem for secure interactions between heterogeneous objects. However, in many real-world use cases, IoT devices are constrained and cannot run natively a full Self-Sovereign Identity stack implementation. For this reason, an edge device has been designed with the capability of supporting a constrained device to create and manage its own self-sovereign identity and it has been developed using Keystone, an open-source framework for building Trusted Execution Environments, providing trust in the system that handles offloaded operations. Such a solution has the advantage to increase the number of devices that can interact in a secure digital ecosystem of this kind and defines a new paradigm called Self-Sovereign Identity as a Service. 

Digital identity makes IoT objects unique and distinguishes them from each other. Self-Sovereign Identity (SSI) aims to provide a digital identity that is both verified and verifiable to IoT objects while building a digital ecosystem for secure interactions between heterogeneous objects. However, in many real-world use cases, IoT devices cannot run natively a full self-sovereign identity stack implementation, due to hardware and software constraints. For this reason, an edge device has been designed with the capability of securely aiding constrained devices to create and manage their own identity according to the SSI paradigm. The software has been developed using Keystone, an open-source framework for building Trusted Execution Environments, for establishing a trusted communication channel between the IoT device and the edge device that handles offloaded operations. By defining a new paradigm called Self-Sovereign Identity as a Service, constrained devices can exploit the full SSI stack on demand. Such a solution has the advantage to increase the number of devices that can interact in a secure digital ecosystem of this kind by shifting the computational operations onto more powerful edge devices. 

\ringraziamenti

{\color{red} TODO}

%% inserire sempre nella tesi per la laurea di I livello, perch? il nome dei tutori non ? indicato sul frontespizio.
%Il lavoro descritto in questa monografia ? stato svolto sotto la supervisione
%del Prof. Antonio Lioy (tutore accademico)% inserire sempre il nome del tutore accademico
% e dell'Ing. Mario Rossi (tutore aziendale)% inserire solo se la monografia ? relativa ad un tirocinio.
%.

%\tablespagetrue % normalmente questa riga non serve ed e' commentata
%\figurespagetrue % normalmente questa riga non serve ed e' commentata

\indici

\mainmatter

\chapter{Introduction}
%\label{chap:1}
%!TEX encoding = IsoLatin
%!TEX main = ../../main.tex

The concept of digital identity has been evolving in the last decades. Digital identity is the expression and storage of one's identity in digital form, which is a set of claims, i.e., assertions of some truths, made about a person or a thing, also named subject. Despite this, only in recent years privacy, control and ownership over digital identity and personal data are being increasingly recognized as relevant factors by individuals \cite{TheLawsOfIdentity}. 

In this document, a solution has been designed to support a constrained device to create and manage its own digital identity. 

contesto
problema
soluzione

% The topics are organized as follows:
% \begin{itemize}
%     \item Chapter \ref{chap:1} 
%     \item Chapter \ref{chap:2} 
%     \item Chapter \ref{chap:3}
%     \item Chapter \ref{chap:4}
%     \item Chapter \ref{chap:5}
% \end{itemize}


%\chapter{Background and Related Work}
%\label{chap:2}
% This chapter provides an overview of the Self-Sovereign-Identity ecosystem and Keystone Enclave framework.
\chapter{Self-Sovereign Identity}
\input{chapters/chapter2-1-ssi.tex}
\chapter{Kestone Enclave}
\input{chapters/chapter2-2-keystone.tex}

%\chapter{Design}
%\label{chap:3}
\chapter{Self-Sovereign Identity as a Service}
%!TEX encoding = IsoLatin
%!TEX main = ../../main.tex



% \chapter{Implementation, setup and tests}
% \label{chap:4}
% %!TEX encoding = IsoLatin
%!TEX main = ../../main.tex

\subsection{Architecture}
{\color{red} ToDo: spiegare provisioning iniziale e architettura finale}
\begin{figure}[!h]
    \centering
    \includesvg[inkscapelatex=false, scale=0.7]{./chapters/images/manufacturer.svg}
    \caption{Manufacturer provisions device with expected hashes, keys and signature of the public key}
    \label{manufacturer-provisioning}
\end{figure}

\begin{figure}[!h]
    \centering
    \includesvg[inkscapelatex=false, scale=0.7]{./chapters/images/Demo architecture3.svg}
    \caption{Demo architecture}
    \label{poc-architecture}
\end{figure}

\chapter{Conclusion and Future Work}
%\label{chap:5}
%!TEX encoding = IsoLatin
%!TEX main = ../../main.tex

In conclusion, this work allows IoT-constrained devices to establish a secure communication channel with the edge device and trusted SSI-related operations computation, which enables to access higher computational performances and more flexible hardware and software requirements. In the future, this opens up new scenarios employing additional and more modern cryptographic primitives for privacy-preserving credentials, which are yet not implemented in the current solution. 
Another future research topic could be to integrate a TLS connection between an external client and the enclave and how securely terminate TLS within the enclave, considering that the untrusted host proxies enclave messages. 
Additionally, another enhancement of the edge device could be to monitor untrusted host integrity with a TPM.
To conclude, this is just an introductory work to the Self-Sovereign Identity as a Service paradigm, which has only just begun being explored by the Cybersecurity team of LINKS Foundation.


\appendix
%This chapters describes how the solution has been implemented and how to use it. 
%to configure the environment used for testing the proposed solution.
\chapter{User Manual}
%!TEX encoding = IsoLatin
%!TEX main = ../../main.tex

\section{Constrained and edge device comparison}
This section describes how to install and execute the code used for testing the cryptographic capabilities of constrained and non-constrained devices.
\subsection{Test MbedTLS library}
Mbed TLS \cite{mbed-tls} is a C library and it was used for testing RSA and ECDSA cryptographic primitives. The version used is \texttt{mbedtls-3.1.0}, available from this repository: \url{https://github.com/Mbed-TLS/mbedtls}. 

\subsubsection{Test on non-constrained device}
First, MbedTLS needs to be installed on the chosen system. \\
\begin{lstlisting}[style=terminal,frame=single]
$ wget 
 https://github.com/Mbed-TLS/mbedtls/archive/refs/tags/v3.1.0.zip
$ unzip v3.1.0.zip
$ cd mbedtls-3.1.0/
$ sudo make install
\end{lstlisting}
Once the library has been installed, enter the Mbed TLS test directory and compile the program.  \\
\begin{lstlisting}[style=terminal,frame=single]
$ cd mbedtls-test
$ gcc -o test main.c mytest.c -lmbedtls -lmbedx509 -lmbedcrypto
\end{lstlisting}
Then tests can be launched and the results can be stored on a file with the following command. \\
\begin{lstlisting}[style=terminal,frame=single]
$ ./test > results.txt
\end{lstlisting}
The output should be similar to the following shown in figure \ref{l-mbedtls-1}. \\
\begin{figure}[H]
\begin{lstlisting}[frame=single]
Seeding the random number generator... ok

Test: ECDSA key gen
p256
0.000266988,
0.000238424,
...
Test: ECDSA sign-gen
p256 key gen time: 0.000260937
0.000004468,0.000309109,0.001057424, # (hash, sign, ver) times
0.000003286,0.000296264,0.001026635,
...

Test: RSA 2048 key gen
0.133808741,
0.087886941,
...

Test: RSA sign-ver
rsa2048 key gen time: 0.065950689
0.000003988,0.002522734,0.000039014 # (hash, sign, ver) times
0.000003206,0.001853610,0.000038684
...
\end{lstlisting}
\caption{Example of MbedTLS tests output on a non-constrained device. \label{l-mbedtls-1}}
\end{figure}
\subsubsection{Test on constrained device}
STM32CubeIDE \cite{cube-ide} has been used (version \texttt{1.9.0}) for developing and flashing the binaries onto the STM32L4+ board \cite{stm32-board-product}.

For simplicity, the complete project is provided and it just needs to be imported into STM32CubeIDE as an existing project. Click \texttt{File > Import > Existing projects into Workspace}, then select the archive file \texttt{mbedtlsv1-stm32.zip} from the file system, select the project \texttt{Test1} and click \texttt{Finish}. Then plug the board into a USB port and click \texttt{Project > Build Project}. The code will be compiled. When it is done (and 0 errors appear in the console panel at the bottom), click \texttt{Run}. 

The output on the board is very similar to the previous one, for debug messages the output of the \texttt{printf} function is redirected to one \texttt{UART} and it can be read with a terminal emulator that supports serial port, such as \texttt{Teraterm} or \texttt{Putty}, by reading the \texttt{COM} that correspond to the \texttt{STM32} debugger. 
The output should be similar to the following shown in figure \ref{l-mbedtls-2}. \\
\begin{figure}[H]
\begin{lstlisting}[frame=single]
Seeding the random number generator... ok
Test: ECDSA key gen, sign-gen
p256
318,
316,
...
key gen time: 319
5;357;1257 # (hash, sign, ver) times
5;361;1264
\end{lstlisting}
\caption{Example of MbedTLS tests output on a constrained device. \label{l-mbedtls-2}}
\end{figure}

%Since it is just a test and some of them take a long time, it is suggested to choose and execute one test at a time by manually comment/uncomment the code in the file \texttt{mytest.c}. 

% \begin{lstlisting}[language=C,frame=single]
% printf("Test: ECDSA key gen, sign-gen\n");
% printf("p256\n");
% test_ECDSA_keygen(MBEDTLS_ECP_DP_SECP256R1, &ctr_drbg, iterations);
% test_ECDSA(MBEDTLS_ECP_DP_SECP256R1, MBEDTLS_MD_SHA256, &ctr_drbg, iterations);
% \end{lstlisting}


\subsection{Test BBS\texttt{+} signatures scheme}
BBS\texttt{+} rust library was used for testing BBS\texttt{+} signatures that can be used to generate signature proofs of knowledge and selective disclosure zero-knowledge proofs. In these tests, only simple single-message signing and verification were tested.

Before starting, we need to install Rust. The installation of Rust is through a tool called Rustup, which is a Rust installer and version management tool. The way to install Rustup differs by platform, on Unix, run the next command in the shell. This downloads and runs \texttt{rustup-init.sh}, which in turn downloads and runs the correct version of the \texttt{rustup-init} executable for your platform. For other platforms check the Rust documentation \cite{rust-install}. \\
\begin{lstlisting}[style=terminal,frame=single]
$ curl https://sh.rustup.rs -sSf | sh
\end{lstlisting}
When Rustup is installed, it will also add the latest stable version of the Rust build tool and package manager, also known as Cargo. The following commands will install dependencies and build the project. \\
\begin{lstlisting}[style=terminal,frame=single]
$ cd bbs-test
$ cargo build
\end{lstlisting}
To launch the test and take executions times of BBS\texttt{+} signatures and verification, run:  \\
\begin{lstlisting}[style=terminal,frame=single]
$ cargo run
\end{lstlisting}
The output should be similar to the following shown in figure \ref{l-bbs}. \\
\begin{figure}[H]
\begin{lstlisting}[frame=single]
AVG Time elapsed in key generation() is: 25 ms
AVG Time elapsed in sign() is: 18 ms
AVG Time elapsed in ver() is: 102 ms
\end{lstlisting}
\caption{Example of BBS\texttt{+} tests output. \label{l-bbs}}
\end{figure}

\section{Proof of concept}

The proof of concept relies on the full Keystone SDK. The easiest way for building and try Keystone and the proof of concept is to use QEMU. QEMU is an open-source machine emulator, in this case, is used to emulate RISC-V architecture.
The proof of concept has been tested with \texttt{Ubuntu 18.04}. 
\subsection{Keystone installation and requirements}
The following dependencies must be installed before installing Keystone. \\
\begin{lstlisting}[style=terminal,frame=single]
$ sudo apt update
$ sudo apt install autoconf automake autotools-dev bc \
  bison build-essential curl expat libexpat1-dev flex gawk \ 
  gcc git gperf libgmp-dev libmpc-dev libmpfr-dev libtool \ 
  texinfo tmux patchutils zlib1g-dev wget bzip2 patch vim-common \
  lbzip2 python pkg-config libglib2.0-dev libpixman-1-dev \
  libssl-dev screen device-tree-compiler expect makeself \
  unzip cpio rsync cmake p7zip-full
\end{lstlisting}
Then check out the Keystone repository and install everything with the quick setup script, it will install the RISC-V toolchain and if \texttt{KEYSTONE\_SDK\_DIR} environment variable is not set, it will also install Keystone SDK. \\
\begin{lstlisting}[style=terminal,frame=single]
$ git clone https://github.com/keystone-enclave/keystone.git
$ cd keystone
$ ./fast-setup.sh
\end{lstlisting}
If everything goes right, the following message is shown: \\
\begin{lstlisting}[frame=single]
    RISC-V toolchain and Keystone SDK have been fully setup
\end{lstlisting}
After running \texttt{fast-setup.sh}, run the following command to temporarily set in the current shell relevant environment variables: \\
\begin{lstlisting}[style=terminal,frame=single]
$ source source.sh
\end{lstlisting}
Otherwise for permanently store the environment variables, if bash is used this command will add the lines in \texttt{source.sh} to the shell's startup file: \\
\begin{lstlisting}[style=terminal,frame=single]
$ cat source.sh >> $HOME/.bashrc
\end{lstlisting}
CMake \cite{cmake} is used as a build system. As \texttt{<build directory>} the name \texttt{build} has been chosen. Then all components can be built, beware that this will take a while. \\
\begin{lstlisting}[style=terminal,frame=single]
$ mkdir build
$ cd build
$ cmake ..
$ make
\end{lstlisting}
\begin{mybox}
\faExclamation\enspace It has been noted that under \texttt{Windows Subsystem for Linux (WSL)} the build of the image can fail. To solve this issue just modify the \texttt{PATH} to not include \texttt{/mnt/c/*} folders.
\end{mybox}

\subsection{Build the demo}
Extract the provided zip file that contains the demo of the proof of concept. The extracted folder should contain the developed code explained in this document. The \texttt{riscv-musl-toolchain} has been used for building the \texttt{server\_eapp}, which can be set with the \texttt{./setup\_musl.sh} script. Then, the \texttt{./quick-start.sh} will build the demo, the script will create two files: \texttt{demo-server.ke} and \texttt{trusted\_client.riscv}. \\
\begin{lstlisting}[style=terminal,frame=single]
$ cd keystone-demo-poc
$ source ./setup_musl.sh
$ SM_HASH=./include/sm_expected_hash.h ./quick-start.sh
\end{lstlisting}
Then once the demo is built, the binaries need to be copied into the Keystone build folder. \\

\begin{lstlisting}[style=terminal,frame=single]
$ cp ./build/demo-server.ke ./build/trusted_client.riscv ../keystone/build/overlay/root/
\end{lstlisting}
Now the QEMU image can be re-generated. \\

\begin{lstlisting}[style=terminal,frame=single]
$ cd keystone/build
$ make image
\end{lstlisting}

\subsection{Run QEMU}
The following script will run QEMU, and start executing from the emulated silicon root of trust. The root of trust then jumps to the SM, and the SM boots Linux. \\
\begin{lstlisting}[style=terminal,frame=single]
$ cd keystone/build
$ ./scripts/run-qemu.sh
\end{lstlisting}
The script will start ssh on a random forwarded localhost port, which allows multiple test runs on the same development machine. The script will print what port it has forwarded ssh to on start. For example, to start a new shell in another window the next command can be used, and it is useful to run the client and the server in two different terminal windows. \\
\begin{lstlisting}[style=terminal,frame=single]
$ ssh root@localhost -p <port number>
\end{lstlisting}
Login as \texttt{\color{RedOrange}root} with the password \texttt{\color{RedOrange}sifive}. You can exit QEMU by \texttt{\color{RedOrange}ctrl-a\ `+`\ x} or using  \texttt{\color{RedOrange}poweroff}  command.

\subsection{Run the demo}
Inside QEMU, run the following commands. The first one will load the Keystone kernel module and the second one will set up the loopback device. \\
\begin{lstlisting}[style=terminal,frame=single]
> insmod keystone-driver.ko 
> ifdown lo && ifup lo           
\end{lstlisting}
On the server side run: \\
\begin{lstlisting}[style=terminal,frame=single]
> ./demo-server.ke         
\end{lstlisting}
On the client side run: \\
\begin{lstlisting}[style=terminal,frame=single]
> ./trusted_client.riscv 127.0.0.1        
\end{lstlisting}
The client will connect to the enclave and perform the remote attestation. If the attestation is successful, the client can send back the session context. Then, if server checks go right, the communication can start and the client can request operation to the server. 

If the enclave server app will be modified, the expected hash values have to be regenerated, otherwise, it can be tested with an option that will ignore the validation of the attestation report (it will still print the status of the validation). \\

\begin{lstlisting}[style=terminal,frame=single]
> ./trusted_client.riscv 127.0.0.1 --ignore-valid        
\end{lstlisting}

\subsection{Expected output}
The output of the client and the server should be similar to the following figures \ref{l-demo-c} and \ref{l-demo-s}.
\begin{figure}[H]
\begin{lstlisting}[frame=single]
              === Security Monitor ===
Hash: e51749130b6036fe85b27409a4ea3e1c078fe4dcb76eb697e7e3cdc5ff
313144628a1ec10558cea5ad94641de7fb31fda759758115caf1144b2c49603f
42db3a
Pubkey: cd98f4a28a8523ba8ecd31175aa0e2330b2f46e7034545254660126a
9f3b8cb9
Signature: e738f4e708f73ffa4a0d3dc2199c9e0ac119bf14b32da33afe842
66803c9ae0766eb6d9f83e6d3b2511cbe8443996feb0bf8714f35c0c678aa593
90846a8d50a
              === Enclave Application ===
Hash: 11ff35526c90c469ca6878dc22494703c554db65406dcc9942417be2d9
010722843d1bda7115604f518fabd47ec3ba79cd89a560847bb5b810c314201e
2217d6
Signature: 58ca380c6b4476ed7b27143d92dec14fff2858ffef8ed6ef40156
515d36da50a220e494d936cafedd23558b8d16ce66c6d5f00bdb7ec973ee46ec
f7a48c6fa0d
Enclave Data: b07dfd3047fb5213b8af9b76594a06891c893001c6c4be448a
d8b13f7eb02a19b27d0263bfd9aa8941345f837788159897a8aea2ecb33da926
b8d4747328eaace08a34d55fcdc41e2938838da173900485e99bcc1fb9eb7b00
dac1e2fe5a5174
                -- Device pubkey --
0faad4ff01178583baa588966f7c1ff32564dd17d7dc2b46cb50a84a69270b4c
[TC] Attestation signature and enclave hash are valid
[TC] Session keys established

        Available services
1. generate keys
2. store verifiable credential
3. get verifiable presentation
 . everything else to quit 
Type the service to request:
> 1
\end{lstlisting}
\caption{Example of output at client side. \label{l-demo-c}}
\end{figure}
\newpage


\begin{figure}[H]
\begin{lstlisting}[frame=single]
Verifying archive integrity... All good.
Uncompressing Keystone Enclave Package
[EH] Got connection from remote client
[SE] NOT USING REAL RANDOMNESS: TEST ONLY
[SE] Stub certificate signature is valid
[SE] [C] Successfully generated session keys.
[EH] Got an encrypted message:
3e5d0575e8bccf73 eb5255dad244eee1 93fcae78f5ad0459 ...

[SE] Sealing key derivation successful!
generate_public_keys

EdDSA_keys_t:  [192]
07FAC4B1522F06D916C55D321BB839F23AEF95AA48E051E53DB2AF284D7845...

Ciphertext  [208]
AB35E2D0F9C3ADFDD10E448D4ABD7A42E00ED87CE569C4AC3809785CC9A028...

Sign  [272]
FA72997814FA1F16D111DA0189D0440F0F245EB606BACACD0566DEB0767D44...
[EH] Saving sealed data
[EH] Filename: /root/075CFECFC6617F ... 26A116B1611_keys_sign
Response payload:  [64]
8E120A739BA94B7E9AE2F590166682170570D7AFCB70026B513405E56E90C1...
\end{lstlisting}
\caption{Example of output at server side. \label{l-demo-s}}
\end{figure}

\subsection{Tools for updating enclave and sm hashes}
If the SM or the eapp will be modified, the expected hashes need to be modified to generate a valid report for the client. The next command will create the necessary files in the \texttt{include} directory. The demo must be recompiled and the QEMU image rebuilt before and after this command is executed. \\

\begin{lstlisting}[style=terminal,frame=single]
# build the demo
# copy the binaries and build qemu image
$ KEYSTONE_BUILD_DIR=<path/to/keystone/build path> ./scripts/get_attestation_modified.sh ./include
# build the demo
# copy the binaries and build qemu image
\end{lstlisting}

If the \texttt{sm\_expected\_hash.h} is not present in the \texttt{include} folder, can be generated for the first time following the instruction in the \texttt{README.md} file of the SM repository available at the following url: \url{https://github.com/keystone-enclave/sm.git}. 

\subsection{Tools for provisioning}

The file \texttt{provision} can be used to generate a new \texttt{test\_client\_key.h}, which will contain the root key pair, the client key pair and the signature of the client public key made by the system administrator with its secrey key. 

It could be modified to meet your purposes (for example to generate more client key pairs and signatures or to use a fixed root key).
The program uses \texttt{libsodium} to generate an Ed25519 key pair for the system administrator. Then will generate another Ed25519 key pair for the client and the public part will be signed with the system administrator key pair. The server will use the system administrator public key to authenticate that the client is part of the system administrator network.
 
Once \texttt{libsodium}  has been installed, compile \texttt{provision.c}, run it and redirect the output to a file with the name \texttt{test\_client\_key.h}. 
The file \texttt{test\_client\_key.h} is already present in the \texttt{include} folder, but it may be necessary to generate a new one and it can be done with the following commands. \\

\begin{lstlisting}[style=terminal,frame=single]
$ git clone https://github.com/jedisct1/libsodium.git
$ cd libsodium
$ git checkout 4917510626c55c1f199ef7383ae164cf96044aea
$ ./configure
$ make && make check
$ sudo make install
$ sudo ldconfig

$ cd keystone-demo/provisioning
$ gcc -o provision provision.c -lsodium

$ ./provision > ../include/test_client_key.h
\end{lstlisting}
\chapter{Developer Manual}
%!TEX encoding = IsoLatin
%!TEX main = ../../main.tex

\section{Constrained and edge device comparison}
\subsection{Test MbedTLS library}
\texttt{mytest.c} is the most relevant file for this subproject. It is equal for both non-constrained devices and the constrained device (STM32L4+ board \cite{stm32-board-product}). The only difference is the function used for measuring the elapsed time of a cryptographic primitive. The function \texttt{clock\_gettime()}, of the standard library \texttt{time.h}, is used on the non-constrained device, while on the STM32L4+ board is used the \texttt{HAL\_GetTick()}. Figure \ref{pseudo-algo} describes in a pseudo-language how to measure the elapsed time of a cryptographic primitive. 
Then, the available tests are described. \\
\begin{figure}[H]
\begin{lstlisting}[language=C,frame=single]
  start = clock_gettime( ) or HAL_GetTick()
  // mbedtls cryptographic primitive
  end = clock_gettime( ) or  HAL_GetTick()
  print ( end - start ) // in ms
\end{lstlisting}
\caption{Pseudo algorithm for measuring the elapsed time. \label{pseudo-algo}}
\end{figure}

\noindent
\texttt{\bfseries int test\_RSA\_keygen(\dots)}\\
It generates \texttt{N} RSA key pair and prints the elapsed for each generation in a CSV format. \\
\textit{Input}:
\begin{itemize}[noitemsep,nolistsep]
  \item \texttt{int key\_size}, size of the generated key (possible values: \texttt{2048}, \texttt{3072} or \texttt{4096})
  \item \texttt{mbedtls\_ctr\_drbg\_context *ctr\_drbg}, reference to \texttt{CTR\_DRBG} context structure for random number generation
  \item \texttt{int iterations}, number of times that the test will be executed
\end{itemize}
\textit{Output}: Returns \texttt{1} if successful, or an \texttt{MBEDTLS\_ERR\_XXX\_XXX} error code \\


\noindent
\texttt{\bfseries int test\_RSA(\dots)}\\
It generates an RSA key pair, computes the RSA signature of a random hashed message (of 1024 bytes) and verifies the generated signature. Then, it prints the elapsed for each signature and verification in a CSV format. (Everything is repeated \texttt{N} times). \\
\textit{Input}:
\begin{itemize}[noitemsep,nolistsep]
  \item \texttt{int key\_size}, size of the generated key (possible values: \texttt{2048}, \texttt{3072} or \texttt{4096})
  \item \texttt{int sha\_alg},
  \item \texttt{mbedtls\_ctr\_drbg\_context *ctr\_drbg}, reference to \texttt{CTR\_DRBG} context structure for random number generation
  \item \texttt{int iterations}, number of times that the test will be executed
\end{itemize}
\textit{Output}:  Returns \texttt{1} if successful, or an \texttt{MBEDTLS\_ERR\_XXX\_XXX} error code \\


\noindent
\texttt{\bfseries int test\_ECDSA\_keygen(\dots)}\\
It generates \texttt{N} EDSA key pair and prints the elapsed for each generation in a CSV format. \\
\textit{Input}:
\begin{itemize}[noitemsep,nolistsep]
  \item \texttt{int ecparams}, the identifier of domain parameters (curve, subgroup and generator),(possible values: \texttt{MBEDTLS\_ECP\_DP\_SECP256R1}, \texttt{MBEDTLS\_ECP\_DP\_SECP384R1} or \texttt{MBEDTLS\_ECP\_DP\_SECP521R1})
  \item \texttt{int sha\_alg},
  \item \texttt{mbedtls\_ctr\_drbg\_context *ctr\_drbg}, reference to \texttt{CTR\_DRBG} context structure for random number generation
  \item \texttt{int iterations}, number of times that the test will be executed
\end{itemize}
\textit{Output}:  Returns \texttt{1} if successful, or an \texttt{MBEDTLS\_ERR\_XXX\_XXX} error code \\


\noindent
\texttt{\bfseries int test\_ECDSA(\dots)}\\
It generates an ECDSA key pair, computes the ECDSA signature of a random hashed message (of 1024 bytes) and verifies the generated signature. Then, it prints the elapsed for each signature and verification in a CSV format. (Everything is repeated \texttt{N} times). \\
\textit{Input}:
\begin{itemize}[noitemsep,nolistsep]
  \item \texttt{int ecparams}, the identifier of domain parameters (curve, subgroup and generator),(possible values: \texttt{MBEDTLS\_ECP\_DP\_SECP256R1}, \texttt{MBEDTLS\_ECP\_DP\_SECP384R1} or \texttt{MBEDTLS\_ECP\_DP\_SECP521R1})
  \item \texttt{int sha\_alg}, digest algorithm (possible values: \texttt{MBEDTLS\_MD\_SHA256}, \texttt{MBEDTLS\_MD\_SHA384} or \texttt{MBEDTLS\_MD\_SHA512})
  \item \texttt{mbedtls\_ctr\_drbg\_context *ctr\_drbg}, reference to \texttt{CTR\_DRBG} context structure for random number generation
  \item \texttt{int iterations}, number of times that the test will be executed
\end{itemize}
\textit{Output}:  Returns \texttt{1} if successful, or an \texttt{MBEDTLS\_ERR\_XXX\_XXX} error code
\newpage
% ===========================================================================
% SUBSECTION
% ===========================================================================

\subsection{Test BBS\texttt{+} signatures scheme}
For this subproject, the most relevant files are:
\begin{itemize}
  \item \texttt{Cargo.toml} where dependencies are added, such as  
  \texttt{bbs} and \texttt{rand}. 
  \item \texttt{main.rs} where tests for BBS\texttt{+} key generation, signature and verification are defined. 
  \begin{itemize} 
    \item \texttt{fn key\_gen\_test(iterations: usize)}
    \item \texttt{fn simple\_sign\_ver\_test(iterations: usize)}
    % \item \texttt{fn pok\_sign\_ver\_test(iterations: usize)}
  \end{itemize}
\end{itemize}

Figure \ref{bbs-pseudo-algo} describes in Rust language how to measure the elapsed time of a cryptographic primitive. \\
\begin{figure}[H]
\begin{lstlisting}[frame=single]
fn <primitive>_test(iterations: usize) {
  let mut sum: u128 = 0;
  for i in 0..iterations {
      let start = Instant::now();
      // bbs+ cryptographic primitive
      let duration = start.elapsed().as_millis();
      sum += duration;
  };
  println!("AVG Time elapsed is: {:?}", sum/iterations as u128);
}
    
\end{lstlisting}
\caption{Pseudo algorithm for measuring the elapsed time.}
\label{bbs-pseudo-algo}
\end{figure}
The program does the following, it launches the test for key generation and the test for signature and verification, operations are executed \texttt{iterations} times and the average is displayed on the output video.  \\
\begin{lstlisting}[frame=single]
fn main() {
  let iterations: usize = 100;
  key_gen_test(iterations);
  simple_sign_ver_test(iterations);
}
\end{lstlisting}


% ===========================================================================
% SUBSECTION
% ===========================================================================
\newpage
\section{Proof of concept}
\subsection{Guide to Keystone Components}
The Keystone repository consists of several sub-components such as gitmodules or directories. This is a brief overview of them. \\
\begin{lstlisting}[frame=single]
+ keystone/
|-- patches/
|  # required patches for submodules
|-- bootrom/
|  # Keystone bootROM for QEMU virt board, including trusted boot chain.
|-- buildroot/
|  # Linux buildroot. Builds a minimal working Linux image for our test platforms.
|-- docs/
|  # Contains read-the-docs formatted and hosted documentation, such as this article.
|-- riscv-gnu-toolchain/
|  # Unmodified toolchain for building riscv targets. Required to build all other components.
|-- linux-keystone-driver/
|  # A loadable kernel module for Keystone enclave.
|-- linux/
|  # Linux kernel
|-- sm/
|  # OpenSBI firmware + Keystone security monitor
|-- qemu/
|  # QEMU
+-- sdk/
  # Tools, libraries, and example apps for building enclaves on Keystone        
\end{lstlisting}
\newpage
\subsection{Guide to Keystone Demo Proof of concept}
The designed solution has been developed starting from the officially \texttt{kestone-demo} repository available at this link: \url{https://github.com/keystone-enclave/keystone-demo}, commit \texttt{8c6c0565e44f3d0e00bc3e4a6e77fc84c9e6d343}. The demo uses test keys and is not safe for production use. \\
\begin{lstlisting}[frame=single]
+ keystone-demo-poc/
|-- docs/
|  # Contains read-the-docs formatted and hosted documentation, such as this article.
|-- include/
|  # Contains shared files between eapp and client
|-- provisioning/
|  # C program to generate new client and root key pairs
|-- scripts/
|  # Contains a script for performing the attestation of sm and eapp
|-- server_eapp/
|  # small enclave server that is capable of remote attestation, secure channel creation, and performing a simple word-counting computation securely
|-- sodium_patches/
|  # Contains patch for libsodium that will run in the server eapp
+-- trusted_client/
  # simple remote client that connects to the host, validates the enclave report, constructs a secure channel, and then can send messages to the host for computation.       
\end{lstlisting}
\newpage
\subsection{Relevant files of the demo}
Below are explained relevant files that change from the official \texttt{kestone-demo} repository.
Functions, types and variables not mentioned here can be deepened in the official \texttt{kestone-demo} repository and Keystone documentation.
\subsubsection{include/eh\_shared.h}
This file is shared between the enclave and the untrusted host. It defines the following data structure for exchanging sealed (encrypted) data between the enclave and the untrusted host. \\

\begin{lstlisting}[language=C,frame=single]
typedef struct stored_data_t{
  unsigned short file_type;
  unsigned char client_pk[crypto_kx_PUBLICKEYBYTES];
  size_t c_len; // content len 
  unsigned char content[]; // Flexible member
} stored_data_t;    
\end{lstlisting}

\subsubsection{include/messages.h}
This file is shared between the client and the enclave and it defines the messages (request and response) they exchange. \\
\begin{lstlisting}[language=C,frame=single]
typedef struct request_message_t {
  unsigned short request_type;
  unsigned char secret[SECRET_LEN];
  size_t len;
  unsigned char payload[]; // Flexible member
} request_message_t;

typedef struct response_message_t {
  unsigned short response_type;
  size_t len;
  unsigned char payload[]; // Flexible member
} response_message_t;
\end{lstlisting}
\newpage
\subsubsection{include/session\_context.c and include/session\_context.h}
The session context is the structure that the client provides to the enclave to prove that is built by a trusted manufacturer. With the function \texttt{session\_context\_from\_buffer} the session context is extracted from the received buffer. \\
\begin{lstlisting}[language=C,frame=single]
struct session_context_t {
  unsigned char  dh_public_key[PUBLIC_KEY_SIZE];
  unsigned char  challenge[CHALLENGE_SIZE];
  unsigned char  data_signature[SIGNATURE_SIZE];
  
  unsigned char  client_public_key[PUBLIC_KEY_SIZE];
  unsigned char  root_signature_of_client_pk[SIGNATURE_SIZE];
};

void session_context_from_buffer(struct session_context_t* session_context, unsigned char* buffer);
\end{lstlisting}

\noindent
\texttt{\bfseries int session\_context\_verify(\dots)}\\
\textit{Input}:
\begin{itemize}[noitemsep,nolistsep]
  \item \texttt{struct session\_context\_t session\_context}, the session context to verify
  \item \texttt{unsigned char* challange}, the challenge that the server has sent to the client
  \item \texttt{const unsigned char* root\_public\_key}, the public key of the manufacturer
\end{itemize}
\textit{Output}: an \texttt{int} value, 1 if it is a valid session context, 0 if not.


% ===========================================================================
% SUBSUBSECTION
% ===========================================================================
\newpage
\subsubsection{server\_eapp/service.c and server\_eapp/service.h}
\label{service.c}
The file \texttt{service.h} just exposes the function that processes the request received by the client. \\   
\begin{lstlisting}[language=C,frame=single]
response_message_t* process_request(request_message_t *request, size_t *pt_finalsize) {

  setup_sealing_material(request->secret);

  switch (request->request_type) {
    case SERVICE_GEN_KEYS:
      return generate_public_keys(pt_finalsize, EdDSA);
      break;
    case SERVICE_STORE_VC:
      return store_verifiable_credential(pt_finalsize, request->payload, request->len);
      break;
    case SERVICE_GET_VP:
      return get_verifiable_presentation(pt_finalsize, request->payload, request->len, EdDSA);
      break;
    default:  
      ocall_print_buffer("Invalid request type!\n");
  }
  return NULL;
}
\end{lstlisting}
% \leavevmode\newline

\noindent
\texttt{\bfseries int store\_data (\dots)}\\
It saves the sealed data in untrusted non-volatile memory. \\
\textit{Input}:
\begin{itemize}[noitemsep,nolistsep]
  \item \texttt{unsigned char* buffer}, data to save
  \item \texttt{size\_t len}, length of the data to save
  \item \texttt{unsigned short file\_type}, type of the data to save (possible values: \texttt{FILE\_CLI\-ENT\_KEYS\_SIGNATURE} or \texttt{FILE\_CLIENT\_VC\_SIGNATURE})
\end{itemize}
\textit{Output}: an \texttt{int} value, 1 if the operation is successful, 0 if not. \\

\noindent
\texttt{\bfseries unsigned char* seal\_data\_and\_sign (\dots)}\\
It encrypts and signs the data to save in untrusted non-volatile memory. \\
\textit{Input}:
\begin{itemize}[noitemsep,nolistsep]
  \item \texttt{unsigned char* data}, data to encrypt and sign
  \item \texttt{size\_t data\_len}, length of the data to encrypt and sign
  \item \texttt{size\_t* sign\_len}, a pointer to store the actual length of the signed message
\end{itemize}
\textit{Output}: the signed message, which includes the signature plus an unaltered copy of the message. \\


\noindent
\texttt{\bfseries response\_message\_t* build\_response (\dots)}\\
It builds the response to sand back to the client. \\
\textit{Input}:
\begin{itemize}[noitemsep,nolistsep]
\item \texttt{size\_t* pt\_finalsize}, pointer to store the final length of the response
\item \texttt{unsigned short response\_type}, type of the response (possible values: \texttt{SERVICE\_\-GEN\_KEYS}, \texttt{SERVICE\_STORE\_VC}, \texttt{SERVICE\_GET\_VP}, \texttt{MSG\_EXIT})
\item \texttt{unsigned char *buffer}, response to send back to the client
\item \texttt{size\_t len}, length of the response
\end{itemize}
\textit{Output}: the built response \\


\noindent
\texttt{\bfseries response\_message\_t* generate\_public\_keys (\dots)}\\
It handles the request of the client to generate two key pairs. It saves the key pairs sealed in the untrusted non-volatile memory and sends back to the client only the public part. \\ 
\textit{Input}:
\begin{itemize}[noitemsep,nolistsep]
\item \texttt{size\_t* pt\_finalsize}, pointer to store the final length of the response
\item \texttt{int key\_type}, the key type that the client chooses to generate 
% (for demo purposes only \texttt{EdDSA} can be generated, in future \texttt{BBS+} signature scheme will be implemented)
\end{itemize}
\textit{Output}: the built response \\

\noindent
\texttt{\bfseries response\_message\_t* store\_verifiable\_credential (\dots)}\\
It handles the request of the client to store a verifiable credential that the client obtained from an issuer. It saves the verifiable credential sealed in the untrusted non-volatile memory and sends back to the client a \texttt{0} if everything goes right. \\
\textit{Input}:
\begin{itemize}[noitemsep,nolistsep]
  \item \texttt{size\_t* pt\_finalsize}, pointer to store the final length of the response
  \item \texttt{unsigned char* vc}, verifiable credential to be stored
  \item \texttt{size\_t vc\_len}, length of the verifiable credential to be stored
\end{itemize}
\textit{Output}: the built response \\

\noindent
\texttt{\bfseries response\_message\_t* get\_verifiable\_presentation (\dots)}\\
It handles the request of the client to generate a verifiable presentation to use when interacting with a verifier. It retrieves from the untrusted non-volatile memory the previously stored verifiable credential and key pairs and sends back to the client the sign of the verifiable credential with one using an assertion key of the type that the client chose. \\
\textit{Input}:
\begin{itemize}[noitemsep,nolistsep]
  \item \texttt{size\_t* pt\_finalsize}, pointer to store the final length of the response
  \item \texttt{unsigned char* nonce}, \textit{(optional)} nonce that the verifier asks the client to insert in the verifiable presentation  
  \item \texttt{size\_t nonce\_len}, length of nonce
  \item \texttt{int key\_type}, the key type that the client previously chose to generate 
  % (for demo purposes only \texttt{EdDSA} can be used, in future \texttt{BBS+} signature scheme will be implemented)
\end{itemize}
\textit{Output}: the built response 

% ===========================================================================
% SUBSUBSECTION
% ===========================================================================
\newpage
\subsubsection{server\_eapp/server\_eapp.c}
In this file, the important functions to mention are:

\noindent
\texttt{\bfseries void attest\_and\_establish\_channel()}\\
In this function, as explained in Chap. \ref{chap:SSIaaS}, the enclave sends its report to the client and waits for the session context of the client. Once received, it validates the session context and established the channel with the client. If some error occurs, the program terminates. \\

\begin{lstlisting}[language=C,frame=single]
void attest_and_establish_channel() {
  // ...  
  randombytes_buf(challenge, CHALLENGE_SIZE);
  // ...  
  ocall_send_report(buffer, MAX_REPORT_SIZE);

  unsigned char session_ctx_buffer[SESSION_CTX_SIZE];
  ocall_wait_for_client_session_ctx(session_ctx_buffer, SESSION_CTX_SIZE);
  // signatures validity check
  validate_session_context(session_ctx_buffer, challenge); 

  channel_establish(); // Ask libsodium to generate session keys based on the received pk
} 
\end{lstlisting}
% \leavevmode\newline
\newpage
\noindent
\texttt{\bfseries void handle\_requests()}\\
In this function, as explained in Chap. \ref{chap:SSIaaS} and in Sect. \ref{service.c}, the enclave will wait for client requests. Whenever a request arrives, the enclave copies the request from the shared region, processes the request (interacting with the enclave host for data sealing), and sends the response back to the client. If some error occurs, the program terminates. \\

\begin{lstlisting}[language=C,frame=single]
void handle_requests() {
  struct edge_data msg;
  while(1){
    ocall_wait_for_request(&msg);
    // ...
    copy_from_shared(request, msg.offset, msg.size);
    // ...
    switch (request->request_type) {
      case MSG_EXIT:
        ocall_print_buffer("Received exit, exiting\n");
        EAPP_RETURN(0);
      break;
      default:
        response = process_request(request, &r_size);
      break;
    }
    if (response == NULL) {
      ocall_print_buffer("Response handling error\n");
      EAPP_RETURN(0);
    }
    // ...
    channel_send((unsigned char*) response, r_size, boxed_buffer);
    ocall_send_reply(boxed_buffer, boxed_size);
    // ... 
  }
}
\end{lstlisting}
    

%\item \texttt{void validate\_session\_context(void* buffer, unsigned char* challange)}



% ===========================================================================
% SUBSUBSECTION
% ===========================================================================
\newpage
\subsubsection{trusted\_client/trusted\_client.c and trusted\_client/trusted\_client.h}
In this file, the important functions to mention are:

\noindent
\texttt{\bfseries int gen\_session\_context(\dots)}\\
It generates the client session context that will contain a data part and the signature of the client's public key made by the system administrator at the provisioning phase. The data part is composed of a key for the Diffie Hellam key exchange protocol and the challenge sent by the server, all the data part is signed by the client's public key. \\
\textit{Input}:
\begin{itemize}[noitemsep,nolistsep]
  \item \texttt{byte* buffer}, a pointer to the buffer that will contain the session context
\end{itemize}
\textit{Output}: an \texttt{int} value, 1 if the operation is successful, 0 if not. \\


\noindent
\texttt{\bfseries request\_message\_t* generate\_request\_message(\dots)}\\
It builds the request message to request a service to the server. \\
\textit{Input}:
\begin{itemize}[noitemsep,nolistsep]
  \item \texttt{char* buffer}, a buffer of the data to send to the server (it can contain the key type or the verifiable credential)
  \item \texttt{size\_t buffer\_len}, length of the data
  \item \texttt{size\_t* finalsize}, a pointer to store the final length of the request message
  \item \texttt{unsigned char* secret}, the secret that the server will use to generate the sealing key in the key derivation function
  \item \texttt{unsigned short request\_type}, the request type that the client has chosen to send to the server (possible values: \texttt{SERVICE\_GEN\_KEYS}, \texttt{SERVICE\_STORE\_VC} or \texttt{SERVICE\_GET\_VP})
\end{itemize}
\textit{Output}: the built request message \\

\noindent
\texttt{\bfseries request\_message\_t* generate\_exit\_message(\dots)}\\
It builds the request message to close the connection with the server. \\
\textit{Input}:
\begin{itemize}[noitemsep,nolistsep]
  \item \texttt{size\_t* finalsize}, a pointer to store the final length of the request message
\end{itemize}
\textit{Output}: the built exit request message


% ===========================================================================
% SUBSECTION
% ===========================================================================

% \begin{lstlisting}[language=C,frame=single]
   
% \end{lstlisting}

% \noindent
% \texttt{}\\
% \textit{Input}:\setlist{nolistsep}
% \begin{itemize}[noitemsep]
%   \item 
% \end{itemize}
% \textit{Output}: 


% bibliografia scritta "a mano"
%% !TEX encoding = IsoLatin

% La bibliografia, da inserirsi solo se ci sono state citazioni.
% In questo caso ricordarsi che bisogna sempre elaborare due volte il file .TEX
% perch� la prima volta viene generata la bibliografia mentre la seconda volta viene inclusa

% NOTA: citare il DOI non � obbligatorio ma MOLTO desiderabile
% NOTE: inserting the DOI is not compulsory bur STRONGLY recommended whenever it exists

\begin{thebibliography}{9} % se ci sono meno di 10 citazioni
%\begin{thebibliography}{99} % se ci sono da 10 a 99 citazioni
%\begin{thebibliography}{999} % se ci sono da 100 a 999 citazioni

% esempio citazione articolo a congresso
% example: reference to a conference paper
\bibitem{psisec}
% autori - authors
I.Enrici, M.Ancilli, A.Lioy,
% titolo articolo - article title
``A psychological approach to information technology security'',
% nome del congresso - conference name
HSI-2010: 3rd Int. Conf. on Human System Interactions,
% luogo (stato) e data del congresso
% town (country) and date of the conference
Rzesz�w (Poland), May 13-15, 2010,
% pagine dell'articolo - article pages
pp.\ 459-466,
% DOI
\doi{10.1109/HSI.2010.5514528}

% esempio citazione articolo su rivista
% example: reference to a journal/magazine article
\bibitem{tpa}
% autori- authors
G.Cabiddu, E.Cesena, R.Sassu, D.Vernizzi, G.Ramunno, A.Lioy,
% titolo dell'articolo -  article title
``Trusted Platform Agent'',
% nome della rivista - name of the journal
IEEE Software,
% volume e numero della rivista (alcune riviste non ce l'hanno)
% volume and issue number (some journals don't have it)
Vol.\ 28, No.\ 2,
% mese e anno di pubblicazione della rivista
% month and year when paper appeared in the journal
March-April 2011,
% pagine dell'articolo  - article pages
pp.\ 35-41,
% DOI
\doi{10.1109/MS.2010.160}


% esempio citazione capitolo di un libro fatto come collezione di contributi da autori diversi
% example: reference to the chapter of a book which is a collection of chapters from different authors
\bibitem{tc}
A.Lioy, G.Ramunno, % autori del capitolo
``Trusted Computing'' % titolo del capitolo
nel libro % in the book
``Handbook of Information and Communication Security'' % titolo del libro
a cura di % edited by
P.Stavroulakis, M.Stamp, % nomi dei curatori
Springer, % nome editore
2010, % anno di pubblicazione
pp.\ 697-717, % pagine del capitolo
\doi{10.1007/978-3-642-04117-4_32}

% esempio citazione pagina web di un progetto
% example: reference to the web page pof a project
\bibitem{openssl}
% nome del progetto - name of the project
The OpenSSL project,
 % URI della pagina web - URI of the web page
\url{http://www.openssl.org/}

% esempio citazione RFC
% example: reference to a RFC
\bibitem{tls12}
T.Dierks, E.Rescorla,
``The Transport Layer Security (TLS) Protocol Version 1.2'',
\rfc{5246}, August 2008,
\doi{10.17487/RFC5246}

% esempio: citazione libro
% example: reference to a book
\bibitem{seceng}
Ross J. Anderson,
``Security engineering'',
Wiley, 2008,
ISBN: 978-0-470-06852-6

\end{thebibliography}


% se la bibliografia ? stata scritta (usando Bibtex) nel file biblio.bib allora commentare la riga precedente e scommentare le due righe seguenti
\bibliographystyle{other/torsec}
\bibliography{other/biblio}

\end{document}
