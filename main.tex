% !TEX encoding = IsoLatin

% La riga soprastante serve per configurare gli editor TeXShop, TeXWorks
% e TeXstudio per gestire questo file con la codifica IsoLatin o Latin 1
% o ISO 8859-1.

% per commentare una riga mettere % al suo inizio
% per s-commentare una riga (ossia attivarla) togliere il % al suo inizio
%
\documentclass[%pdfa% formato PDF/A, obbligatorio per l'archiviazione delle tesi di Polito
%,cucitura%lascia margine per la rilegatura
,twoside% per stampa fronte-retro (fortemente consigliato per tesi voluminose, opzionale per le altre)
%,12pt% font pi? grande (12pt) rispetto a quello normalmente usato (11pt)
]{toptesi}
%
\usepackage{hyperref}
\hypersetup{%
    pdfpagemode={UseOutlines},
    bookmarksopen,
    pdfstartview={FitH},
    colorlinks,
    linkcolor={blue},
    citecolor={red},
    urlcolor={blue}
  }
% \documentclass[11pt,twoside,oldstyle,autoretitolo,classica,greek]{toptesi}
% \usepackage[or]{teubner}
%%%%%%%%%%%%%%%%%%%%%%%%%%%%%%%%%%%%%%%%%%%%%%%%%%%%
%
% Esempio di composizione di tesi di laurea.
%
% Questo esempio e' stato preparato inizialmente 13-marzo-1989
% e poi e' stato modificato via via che TOPtesi andava
% arricchendosi di altre possibilita'.
%
% Nel seguito laurea "quinquennale" sta anche per "specialistica" o "magistrale"

% Cambiare encoding a piacere; oppure non caricare nessun encoding se si usano
% solo caratteri a 7 bit (ASCII) nei file d'entrata.
%
\usepackage[latin1]{inputenc}% IMPORTANTE! usare codifica ISO-8859-1 per le lettere accentate


% !TEX encoding = IsoLatin

% per inserire uno spazio "fantasma" nella definizione di un'abbreviazione
\usepackage{xspace}

% per inserire un DOI senza problemi coi caratteri "strani" ivi presenti
\usepackage{doi}
\renewcommand{\doitext}{DOI }% originally was "doi:"

% per inserire correttamente le unit� di misura SI (incluse quelle binarie)
\usepackage[binary-units]{siunitx}
% se si desidera usare / invece che la potenza -1 per indicare "al secondo"
\sisetup{per-mode=symbol}

% per inserire codice di programmazione complesso
\usepackage{listings}% per inserire codice di programmazione complesso
\lstset{
basicstyle=\ttfamily,
columns=fullflexible,
xleftmargin=3ex,
breaklines,
breakatwhitespace,
escapechar=`
}

% modify some page parameters
\setlength{\parskip}{\medskipamount}

% riga orizzontale
\newcommand{\HRule}{\rule{\linewidth}{0.2mm}}
% esempio di creazione di semplici abbreviazioni
\newcommand{\ltx}{\LaTeX\xspace}
\newcommand{\txw}{TeXworks\xspace}
\newcommand{\mik}{MikTex\xspace}
\newcommand{\html}{HTML\xspace}
\newcommand{\xhtml}{XHTML\xspace}

% esempio di creazione di un'abbreviazione con un parametro (il cui uso � indicato da #1)
\newcommand{\cmd}[1]{\texttt{#1}\xspace}
% per citare un RFC, es. \rfc{822}
\newcommand{\rfc}[1]{RFC-#1\xspace}
% per citare un file (es. \file{autoexec.bat}) o una URI fittizia (es. \file{http://www.lioy.it/})
% per le URI vere usare \url o \href
\newcommand{\file}[1]{\texttt{#1}\xspace}
% per inserire codice di esempio in-line
\newcommand{\code}[1]{\lstinline|#1|}
% importante per i pathname Windows perch� non si pu� usare \ essendo un carattere riservato di Latex
\newcommand{\bs}{\textbackslash}
% definizione di un termine: formattazione ed inserimento nell'indice
\newcommand{\tdef}[1]{\textit{#1}\index{#1}}
% meta-termine, usato tipicamente nelle definizioni dei tag
\newcommand{\meta}[1]{\textit{#1}}
% abbreviazioni in inglese
\newcommand{\ie}{i.e.\xspace}
\newcommand{\eg}{e.g.\xspace}

\begin{document}
\selectlanguage{english}


\ateneo{Politecnico di Torino}

%%% scegliere la propria facolt? (solo PRIMA dell'AA 2012-2013)
%
%\facolta[III]{Ingegneria dell'Informazione}
%\facolta[IV]{Organizzazione d'Impresa\\e Ingegneria Gestionale}
%\Materia{Remote sensing}% uso sconsigliato

%\monografia{Gestione informatizzata di un magazzino ricambi}% per la laurea triennale
\titolo{Self-Sovereign-Identity as a Service}% per la laurea quinquennale e il dottorato
\sottotitolo{Computation offloading in TEEs for IoT constrained devices}% NON obbligatorio, per la laurea quinquennale e il dottorato

%%% scegliere il proprio corso
%
%\corsodilaurea{Ingegneria dell'Organizzazione d'Impresa}% per la laurea di primo e secondo livello
%\corsodilaurea{Ingegneria Logistica e della Produzione}% per la laurea di primo e secondo livello
%\corsodilaurea{Ingegneria Gestionale}% per la laurea di primo e secondo livello
\corsodilaurea{Computer Engineering}% per la laurea di primo e secondo livello
%\corsodidottorato{Meccanica}% per il dottorato

\candidato{Luca \textsc{Giorgino}}% per tutti i percorsi
%\secondocandidato{Evangelista \textsc{Torricelli}}% per la laurea magistrale solamente
%\direttore{prof. Albert Einstein}% per il dottorato
%\coordinatore{prof. Albert Einstein}% per il dottorato
\relatore{prof.\ Antonio Lioy}% per la laurea e il dottorato
%\secondorelatore{dipl.~ing.~Werner von Braun}% per la laurea magistrale
%\terzorelatore{{\tabular{@{}l}dott.\ Neil Armstrong\\prof. Maria Rossi\endtabular}}% per la laurea magistrale
%\tutore{ing.~Karl Von Braun}% per il dottorato
%\tutoreaziendale{PhD.\ ing.\ Andrea Vesco} % solo per la laurea di secondo livello con tesi svolta in azienda
\tutoreaziendale{PhD.\ ing.\ Andrea Vesco\\PhD.\ ing.\ Alberto Carelli} % solo per la laurea di secondo livello con tesi svolta in azienda
% \NomeTutoreAziendale{Supervisore aziendale\\LINKS FOUNDATION}
%\sedutadilaurea{Ottobre 2022}% per la laurea quinquennale
%\esamedidottorato{Novembre 1610}% per il dottorato
%\sedutadilaurea{\textsc{Ottobre} 2022}% per la laurea triennale
\sedutadilaurea{\textsc{Academic~year} 2021-2022}% per la laurea magistrale
%\annoaccademico{1615-1616}% solo con l'opzione classica
%\annoaccademico{2006-2007}% idem
%\ciclodidottorato{XV}% per il dottorato
\logosede{other/logopolito}
%
%\chapterbib %solo per vedere che cosa succede; e' preferibile comporre una sola bibliografia
%\AdvisorName{Supervisors}
%\newtheorem{osservazione}{Osservazione}% Standard LaTeX

%\usepackage[a-1b]{pdfx}
%\hypersetup{%
%    pdfpagemode={UseOutlines},
%    bookmarksopen,
%    pdfstartview={FitH},
%    colorlinks,
%    linkcolor={blue},
%    citecolor={green},
%    urlcolor={blue}
%  }

%
% per numerare e far comparire nell'indice anche le sezioni di quarto livello
% SCONSIGLIATO! da usarsi solo in caso di estrema necessit?
%\setcounter{secnumdepth}{4}% section-numbering-depth
%\setcounter{tocdepth}{4}% TOC-numbering-depth (TOC=Table-Of-Content)

%\setbindingcorrection{3mm}

\errorcontextlines=9

\iflanguage{english}{
  %\retrofrontespizio{This work is subject to the Creative Commons Licence}
  \DottoratoIn{PhD Course in\space}
  \CorsoDiLaureaIn{Master degree course in\space}
  \NomeMonografia{Bachelor Degree Final Work}
  \TesiDiLaurea{Master Degree Thesis}
  \NomeDissertazione{PhD Dissertation}
  \InName{in}
  \CandidateName{Candidate}
  \AdvisorName{Supervisor}
  \TutorName{Tutor}
  \NomeTutoreAziendale{Tutor\\LINKS FOUNDATION}
  \CycleName{cycle}
  \NomePrimoTomo{First volume}
  \NomeSecondoTomo{Second Volume}
  \NomeTerzoTomo{Third Volume}
  \NomeQuartoTomo{Fourth Volume}
}{}

\frontespizio
\paginavuota
\newpage
%per sfruttare meglio lo spazio nella pagina
\advance\voffset -5mm
\advance\textheight 30mm

% opzionale, solo se si vuole dedicare la tesi a delle persone care
\begin{dedica}
Una piccola dedica
\end{dedica}

\sommario

Inserire qui un breve sommario della tesi.

\ringraziamenti

Opzionali, solo nel caso si sia ricevuto un aiuto speciale e particolarmente rilevante.

% Questo lavoro non sarebbe stato possibile senza la supervisione ed i consigli dei miei relatori, il
% Prof. Antonio Lioy e il Prof. Andrea Atzeni.
% Vorrei ringraziare inoltre l?assistente di laboratorio e ricercatore Ignazio Pedone per avermi
% concesso l?utilizzo di una macchina virtuale del Politecnico di Torino al fine di rendere disponibile
% al pubblico il gioco realizzato.
% Ringrazio la mia famiglia, che mi ha supportato lungo tutto il percorso universitario, anche
% nei momenti pi`u difficili. Ringrazio in particolare mio padre Ezio e mia madre Paola che mi hanno
% permesso di studiare al Politecnico di Torino.
% Infine un ringraziamento speciale va a Chiara, che mi ha sempre mostrato il suo supporto e il
% suo amore in questi anni di studio

%% inserire sempre nella tesi per la laurea di I livello, perch? il nome dei tutori non ? indicato sul frontespizio.
%Il lavoro descritto in questa monografia ? stato svolto sotto la supervisione
%del Prof. Antonio Lioy (tutore accademico)% inserire sempre il nome del tutore accademico
% e dell'Ing. Mario Rossi (tutore aziendale)% inserire solo se la monografia ? relativa ad un tirocinio.
%.

%\tablespagetrue % normalmente questa riga non serve ed e' commentata
%\figurespagetrue % normalmente questa riga non serve ed e' commentata

\indici

\mainmatter

\chapter{Introduction}
\chapter{Background and related work}

\section{Struttura della tesi}

Normalmente la tesi per la laurea magistrale � composta da almeno cinque capitoli.

Il primo capitolo contiene l'introduzione e descrive il contesto o il problema da cui � scaturito l'argomento della tesi.
In altre parole, fornisce una giustificazione per lo svolgimento della tesi e dovrebbe incuriosire il lettore a scoprire come il problema � stato affrontato e risolto.

Nel caso di una tesi svolta in azienda � possibile inserire anche una breve descrizione dell'azienda o ente presso cui � stata svolta, con particolare riferimento al contesto applicativo, ingegneristico o commerciale da cui � nata la necessit� alla base della tesi.

Il secondo capitolo fornisce -- se necessario -- un'analisi critica dei lavori precedenti sul tema trattato nella tesi.

Il terzo capitolo descrive l'approccio usato per risolvere il problema. Ad esempio, nel caso di progettazione di un'applicazione web, si possono fornire le specifiche delle singole pagine, il collegamento tra le varie pagine, la struttura del database e la logica pplicativa.
Questo capitolo pu� anche essere spezzato in pi� capitoli se dovesse risultare particolarmente voluminoso, dedicando ciascun capitolo ad uno specifico tema (es. la progettazione del database, la logica applicativa).

Il capitolo successivo fornisce i risultati dell'attivit� svolta, cercando di misurare i risultati in modo quantitativo.

Infine il capitolo conclusivo riassume i risultati conseguiti ed indica se ci sono ulteriori possibili sviluppi o miglioramenti.

\selectlanguage{english}
\section{Writing the thesis in English}

If the thesis is written in English, then some terms should be changed (such Capitolo that should become Chapter) and hyphenation should follow the rules of the English language.

To achieve this effect, please insert \verb+\selectlanguage{english}+ in the preamble
(that is, before \verb+\begin{document}+) and do not forget to select the UK-English dictionary for the spell checker.


\chapter{SSI as a Service}

%!TEX encoding = IsoLatin
%!TEX main = ../../main.tex

\section{IoT constrained devices}
\section{Use cases analiys}
\section{Test and result analysis}


\chapter{Design and Implementation}

%!TEX encoding = IsoLatin
%!TEX main = ../../main.tex
\section{Trusted Execution Environments (TEEs)}
General description
\section{Keystone enclave}
framework explaination



\chapter{Setup, test and result analysis}


\chapter{Conclusion and future work}

Qui si inseriscono brevi conclusioni sul lavoro svolto, senza ripetere inutilmente il sommario.

Si possono evidenziare i punti di forza e quelli di debolezza, nonch? i possibili sviluppi futuri o attivit? da svolgere per migliorare i risultati.

% bibliografia scritta "a mano"
%% !TEX encoding = IsoLatin

% La bibliografia, da inserirsi solo se ci sono state citazioni.
% In questo caso ricordarsi che bisogna sempre elaborare due volte il file .TEX
% perch� la prima volta viene generata la bibliografia mentre la seconda volta viene inclusa

% NOTA: citare il DOI non � obbligatorio ma MOLTO desiderabile
% NOTE: inserting the DOI is not compulsory bur STRONGLY recommended whenever it exists

\begin{thebibliography}{9} % se ci sono meno di 10 citazioni
%\begin{thebibliography}{99} % se ci sono da 10 a 99 citazioni
%\begin{thebibliography}{999} % se ci sono da 100 a 999 citazioni

% esempio citazione articolo a congresso
% example: reference to a conference paper
\bibitem{psisec}
% autori - authors
I.Enrici, M.Ancilli, A.Lioy,
% titolo articolo - article title
``A psychological approach to information technology security'',
% nome del congresso - conference name
HSI-2010: 3rd Int. Conf. on Human System Interactions,
% luogo (stato) e data del congresso
% town (country) and date of the conference
Rzesz�w (Poland), May 13-15, 2010,
% pagine dell'articolo - article pages
pp.\ 459-466,
% DOI
\doi{10.1109/HSI.2010.5514528}

% esempio citazione articolo su rivista
% example: reference to a journal/magazine article
\bibitem{tpa}
% autori- authors
G.Cabiddu, E.Cesena, R.Sassu, D.Vernizzi, G.Ramunno, A.Lioy,
% titolo dell'articolo -  article title
``Trusted Platform Agent'',
% nome della rivista - name of the journal
IEEE Software,
% volume e numero della rivista (alcune riviste non ce l'hanno)
% volume and issue number (some journals don't have it)
Vol.\ 28, No.\ 2,
% mese e anno di pubblicazione della rivista
% month and year when paper appeared in the journal
March-April 2011,
% pagine dell'articolo  - article pages
pp.\ 35-41,
% DOI
\doi{10.1109/MS.2010.160}


% esempio citazione capitolo di un libro fatto come collezione di contributi da autori diversi
% example: reference to the chapter of a book which is a collection of chapters from different authors
\bibitem{tc}
A.Lioy, G.Ramunno, % autori del capitolo
``Trusted Computing'' % titolo del capitolo
nel libro % in the book
``Handbook of Information and Communication Security'' % titolo del libro
a cura di % edited by
P.Stavroulakis, M.Stamp, % nomi dei curatori
Springer, % nome editore
2010, % anno di pubblicazione
pp.\ 697-717, % pagine del capitolo
\doi{10.1007/978-3-642-04117-4_32}

% esempio citazione pagina web di un progetto
% example: reference to the web page pof a project
\bibitem{openssl}
% nome del progetto - name of the project
The OpenSSL project,
 % URI della pagina web - URI of the web page
\url{http://www.openssl.org/}

% esempio citazione RFC
% example: reference to a RFC
\bibitem{tls12}
T.Dierks, E.Rescorla,
``The Transport Layer Security (TLS) Protocol Version 1.2'',
\rfc{5246}, August 2008,
\doi{10.17487/RFC5246}

% esempio: citazione libro
% example: reference to a book
\bibitem{seceng}
Ross J. Anderson,
``Security engineering'',
Wiley, 2008,
ISBN: 978-0-470-06852-6

\end{thebibliography}


% se la bibliografia ? stata scritta (usando Bibtex) nel file biblio.bib allora commentare la riga precedente e scommentare le due righe seguenti
\bibliographystyle{other/torsec}
\bibliography{other/biblio2}

\appendix
\chapter{ST board and server test installation}
\chapter{Keystone and demo install}
\end{document}
