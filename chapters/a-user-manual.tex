%!TEX encoding = IsoLatin
%!TEX main = ../../main.tex

\section{Constrained and edge device comparison}
This section describes how to install and execute the code used for testing the cryptographic capabilities of constrained and non-constrained devices.
\subsection{Test MbedTLS library}
Mbed TLS \cite{mbed-tls} is a C library and it was used for testing RSA and ECDSA cryptographic primitives. The version used is \texttt{mbedtls-3.1.0}, available from this repository: \url{https://github.com/Mbed-TLS/mbedtls}. 

\subsubsection{Test on non-constrained device}
First, MbedTLS needs to be installed on the chosen system. \\
\begin{lstlisting}[style=terminal,frame=single]
$ wget 
 https://github.com/Mbed-TLS/mbedtls/archive/refs/tags/v3.1.0.zip
$ unzip v3.1.0.zip
$ cd mbedtls-3.1.0/
$ sudo make install
\end{lstlisting}
Once the library has been installed, enter the Mbed TLS test directory and compile the program.  \\
\begin{lstlisting}[style=terminal,frame=single]
$ cd mbedtls-test
$ gcc -o test main.c mytest.c -lmbedtls -lmbedx509 -lmbedcrypto
\end{lstlisting}
Then tests can be launched and the results can be stored on a file with the following command. \\
\begin{lstlisting}[style=terminal,frame=single]
$ ./test > results.txt
\end{lstlisting}
The output should be similar to the following shown in figure \ref{l-mbedtls-1}. \\
\begin{figure}[H]
\begin{lstlisting}[frame=single]
Seeding the random number generator... ok

Test: ECDSA key gen
p256
0.000266988,
0.000238424,
...
Test: ECDSA sign-gen
p256 key gen time: 0.000260937
0.000004468,0.000309109,0.001057424, # (hash, sign, ver) times
0.000003286,0.000296264,0.001026635,
...

Test: RSA 2048 key gen
0.133808741,
0.087886941,
...

Test: RSA sign-ver
rsa2048 key gen time: 0.065950689
0.000003988,0.002522734,0.000039014 # (hash, sign, ver) times
0.000003206,0.001853610,0.000038684
...
\end{lstlisting}
\caption{Example of MbedTLS tests output on a non-constrained device. \label{l-mbedtls-1}}
\end{figure}
\subsubsection{Test on constrained device}
STM32CubeIDE \cite{cube-ide} has been used (version \texttt{1.9.0}) for developing and flashing the binaries onto the STM32L4+ board \cite{stm32-board-product}.

For simplicity, the complete project is provided and it just needs to be imported into STM32CubeIDE as an existing project. Click \texttt{File > Import > Existing projects into Workspace}, then select the archive file \texttt{mbedtlsv1-stm32.zip} from the file system, select the project \texttt{Test1} and click \texttt{Finish}. Then plug the board into a USB port and click \texttt{Project > Build Project}. The code will be compiled. When it is done (and 0 errors appear in the console panel at the bottom), click \texttt{Run}. 

The output on the board is very similar to the previous one, for debug messages the output of the \texttt{printf} function is redirected to one \texttt{UART} and it can be read with a terminal emulator that supports serial port, such as \texttt{Teraterm} or \texttt{Putty}, by reading the \texttt{COM} that correspond to the \texttt{STM32} debugger. 
The output should be similar to the following shown in figure \ref{l-mbedtls-2}. \\
\begin{figure}[H]
\begin{lstlisting}[frame=single]
Seeding the random number generator... ok
Test: ECDSA key gen, sign-gen
p256
318,
316,
...
key gen time: 319
5;357;1257 # (hash, sign, ver) times
5;361;1264
\end{lstlisting}
\caption{Example of MbedTLS tests output on a constrained device. \label{l-mbedtls-2}}
\end{figure}

%Since it is just a test and some of them take a long time, it is suggested to choose and execute one test at a time by manually comment/uncomment the code in the file \texttt{mytest.c}. 

% \begin{lstlisting}[language=C,frame=single]
% printf("Test: ECDSA key gen, sign-gen\n");
% printf("p256\n");
% test_ECDSA_keygen(MBEDTLS_ECP_DP_SECP256R1, &ctr_drbg, iterations);
% test_ECDSA(MBEDTLS_ECP_DP_SECP256R1, MBEDTLS_MD_SHA256, &ctr_drbg, iterations);
% \end{lstlisting}


\subsection{Test BBS\texttt{+} signatures scheme}
BBS\texttt{+} rust library was used for testing BBS\texttt{+} signatures that can be used to generate signature proofs of knowledge and selective disclosure zero-knowledge proofs. In these tests, only simple single-message signing and verification were tested.

Before starting, we need to install Rust. The installation of Rust is through a tool called Rustup, which is a Rust installer and version management tool. The way to install Rustup differs by platform, on Unix, run the next command in the shell. This downloads and runs \texttt{rustup-init.sh}, which in turn downloads and runs the correct version of the \texttt{rustup-init} executable for your platform. For other platforms check the Rust documentation \cite{rust-install}. \\
\begin{lstlisting}[style=terminal,frame=single]
$ curl https://sh.rustup.rs -sSf | sh
\end{lstlisting}
When Rustup is installed, it will also add the latest stable version of the Rust build tool and package manager, also known as Cargo. The following commands will install dependencies and build the project. \\
\begin{lstlisting}[style=terminal,frame=single]
$ cd bbs-test
$ cargo build
\end{lstlisting}
To launch the test and take executions times of BBS\texttt{+} signatures and verification, run:  \\
\begin{lstlisting}[style=terminal,frame=single]
$ cargo run
\end{lstlisting}
The output should be similar to the following shown in figure \ref{l-bbs}. \\
\begin{figure}[H]
\begin{lstlisting}[frame=single]
AVG Time elapsed in key generation() is: 25 ms
AVG Time elapsed in sign() is: 18 ms
AVG Time elapsed in ver() is: 102 ms
\end{lstlisting}
\caption{Example of BBS\texttt{+} tests output. \label{l-bbs}}
\end{figure}

\section{Proof of concept}

The proof of concept relies on the full Keystone SDK. The easiest way for building and try Keystone and the proof of concept is to use QEMU. QEMU is an open-source machine emulator, in this case, is used to emulate RISC-V architecture.
The proof of concept has been tested with \texttt{Ubuntu 18.04}. 
\subsection{Keystone installation and requirements}
The following dependencies must be installed before installing Keystone. \\
\begin{lstlisting}[style=terminal,frame=single]
$ sudo apt update
$ sudo apt install autoconf automake autotools-dev bc \
  bison build-essential curl expat libexpat1-dev flex gawk \ 
  gcc git gperf libgmp-dev libmpc-dev libmpfr-dev libtool \ 
  texinfo tmux patchutils zlib1g-dev wget bzip2 patch vim-common \
  lbzip2 python pkg-config libglib2.0-dev libpixman-1-dev \
  libssl-dev screen device-tree-compiler expect makeself \
  unzip cpio rsync cmake p7zip-full
\end{lstlisting}
Then check out the Keystone repository and install everything with the quick setup script, it will install the RISC-V toolchain and if \texttt{KEYSTONE\_SDK\_DIR} environment variable is not set, it will also install Keystone SDK. \\
\begin{lstlisting}[style=terminal,frame=single]
$ git clone https://github.com/keystone-enclave/keystone.git
$ cd keystone
$ ./fast-setup.sh
\end{lstlisting}
If everything goes right, the following message is shown: \\
\begin{lstlisting}[frame=single]
    RISC-V toolchain and Keystone SDK have been fully setup
\end{lstlisting}
After running \texttt{fast-setup.sh}, run the following command to temporarily set in the current shell relevant environment variables: \\
\begin{lstlisting}[style=terminal,frame=single]
$ source source.sh
\end{lstlisting}
Otherwise for permanently store the environment variables, if bash is used this command will add the lines in \texttt{source.sh} to the shell's startup file: \\
\begin{lstlisting}[style=terminal,frame=single]
$ cat source.sh >> $HOME/.bashrc
\end{lstlisting}
CMake \cite{cmake} is used as a build system. As \texttt{<build directory>} the name \texttt{build} has been chosen. Then all components can be built, beware that this will take a while. \\
\begin{lstlisting}[style=terminal,frame=single]
$ mkdir build
$ cd build
$ cmake ..
$ make
\end{lstlisting}
\begin{mybox}
\faExclamation\enspace It has been noted that under \texttt{Windows Subsystem for Linux (WSL)} the build of the image can fail. To solve this issue just modify the \texttt{PATH} to not include \texttt{/mnt/c/*} folders.
\end{mybox}

\subsection{Build the demo}
Extract the provided zip file that contains the demo of the proof of concept. The extracted folder should contain the developed code explained in this document. The \texttt{riscv-musl-toolchain} has been used for building the \texttt{server\_eapp}, which can be set with the \texttt{./setup\_musl.sh} script. Then, the \texttt{./quick-start.sh} will build the demo, the script will create two files: \texttt{demo-server.ke} and \texttt{trusted\_client.riscv}. \\
\begin{lstlisting}[style=terminal,frame=single]
$ cd keystone-demo-poc
$ source ./setup_musl.sh
$ SM_HASH=./include/sm_expected_hash.h ./quick-start.sh
\end{lstlisting}
Then once the demo is built, the binaries need to be copied into the Keystone build folder. \\

\begin{lstlisting}[style=terminal,frame=single]
$ cp ./build/demo-server.ke ./build/trusted_client.riscv ../keystone/build/overlay/root/
\end{lstlisting}
Now the QEMU image can be re-generated. \\

\begin{lstlisting}[style=terminal,frame=single]
$ cd keystone/build
$ make image
\end{lstlisting}

\subsection{Run QEMU}
The following script will run QEMU, and start executing from the emulated silicon root of trust. The root of trust then jumps to the SM, and the SM boots Linux. \\
\begin{lstlisting}[style=terminal,frame=single]
$ cd keystone/build
$ ./scripts/run-qemu.sh
\end{lstlisting}
The script will start ssh on a random forwarded localhost port, which allows multiple test runs on the same development machine. The script will print what port it has forwarded ssh to on start. For example, to start a new shell in another window the next command can be used, and it is useful to run the client and the server in two different terminal windows. \\
\begin{lstlisting}[style=terminal,frame=single]
$ ssh root@localhost -p <port number>
\end{lstlisting}
Login as \texttt{\color{RedOrange}root} with the password \texttt{\color{RedOrange}sifive}. You can exit QEMU by \texttt{\color{RedOrange}ctrl-a\ `+`\ x} or using  \texttt{\color{RedOrange}poweroff}  command.

\subsection{Run the demo}
Inside QEMU, run the following commands. The first one will load the Keystone kernel module and the second one will set up the loopback device. \\
\begin{lstlisting}[style=terminal,frame=single]
> insmod keystone-driver.ko 
> ifdown lo && ifup lo           
\end{lstlisting}
On the server side run: \\
\begin{lstlisting}[style=terminal,frame=single]
> ./demo-server.ke         
\end{lstlisting}
On the client side run: \\
\begin{lstlisting}[style=terminal,frame=single]
> ./trusted_client.riscv 127.0.0.1        
\end{lstlisting}
The client will connect to the enclave and perform the remote attestation. If the attestation is successful, the client can send back the session context. Then, if server checks go right, the communication can start and the client can request operation to the server. 

If the enclave server app will be modified, the expected hash values have to be regenerated, otherwise, it can be tested with an option that will ignore the validation of the attestation report (it will still print the status of the validation). \\

\begin{lstlisting}[style=terminal,frame=single]
> ./trusted_client.riscv 127.0.0.1 --ignore-valid        
\end{lstlisting}

\subsection{Expected output}
The output of the client and the server should be similar to the following figures \ref{l-demo-c} and \ref{l-demo-s}.
\begin{figure}[H]
\begin{lstlisting}[frame=single]
              === Security Monitor ===
Hash: e51749130b6036fe85b27409a4ea3e1c078fe4dcb76eb697e7e3cdc5ff
313144628a1ec10558cea5ad94641de7fb31fda759758115caf1144b2c49603f
42db3a
Pubkey: cd98f4a28a8523ba8ecd31175aa0e2330b2f46e7034545254660126a
9f3b8cb9
Signature: e738f4e708f73ffa4a0d3dc2199c9e0ac119bf14b32da33afe842
66803c9ae0766eb6d9f83e6d3b2511cbe8443996feb0bf8714f35c0c678aa593
90846a8d50a
              === Enclave Application ===
Hash: 11ff35526c90c469ca6878dc22494703c554db65406dcc9942417be2d9
010722843d1bda7115604f518fabd47ec3ba79cd89a560847bb5b810c314201e
2217d6
Signature: 58ca380c6b4476ed7b27143d92dec14fff2858ffef8ed6ef40156
515d36da50a220e494d936cafedd23558b8d16ce66c6d5f00bdb7ec973ee46ec
f7a48c6fa0d
Enclave Data: b07dfd3047fb5213b8af9b76594a06891c893001c6c4be448a
d8b13f7eb02a19b27d0263bfd9aa8941345f837788159897a8aea2ecb33da926
b8d4747328eaace08a34d55fcdc41e2938838da173900485e99bcc1fb9eb7b00
dac1e2fe5a5174
                -- Device pubkey --
0faad4ff01178583baa588966f7c1ff32564dd17d7dc2b46cb50a84a69270b4c
[TC] Attestation signature and enclave hash are valid
[TC] Session keys established

        Available services
1. generate keys
2. store verifiable credential
3. get verifiable presentation
 . everything else to quit 
Type the service to request:
> 1
\end{lstlisting}
\caption{Example of output at client side. \label{l-demo-c}}
\end{figure}
\newpage


\begin{figure}[H]
\begin{lstlisting}[frame=single]
Verifying archive integrity... All good.
Uncompressing Keystone Enclave Package
[EH] Got connection from remote client
[SE] NOT USING REAL RANDOMNESS: TEST ONLY
[SE] Stub certificate signature is valid
[SE] [C] Successfully generated session keys.
[EH] Got an encrypted message:
3e5d0575e8bccf73 eb5255dad244eee1 93fcae78f5ad0459 ...

[SE] Sealing key derivation successful!
generate_public_keys

EdDSA_keys_t:  [192]
07FAC4B1522F06D916C55D321BB839F23AEF95AA48E051E53DB2AF284D7845...

Ciphertext  [208]
AB35E2D0F9C3ADFDD10E448D4ABD7A42E00ED87CE569C4AC3809785CC9A028...

Sign  [272]
FA72997814FA1F16D111DA0189D0440F0F245EB606BACACD0566DEB0767D44...
[EH] Saving sealed data
[EH] Filename: /root/075CFECFC6617F ... 26A116B1611_keys_sign
Response payload:  [64]
8E120A739BA94B7E9AE2F590166682170570D7AFCB70026B513405E56E90C1...
\end{lstlisting}
\caption{Example of output at server side. \label{l-demo-s}}
\end{figure}

\subsection{Tools for updating enclave and sm hashes}
If the SM or the eapp will be modified, the expected hashes need to be modified to generate a valid report for the client. The next command will create the necessary files in the \texttt{include} directory. The demo must be recompiled and the QEMU image rebuilt before and after this command is executed. \\

\begin{lstlisting}[style=terminal,frame=single]
# build the demo
# copy the binaries and build qemu image
$ KEYSTONE_BUILD_DIR=<path/to/keystone/build path> ./scripts/get_attestation_modified.sh ./include
# build the demo
# copy the binaries and build qemu image
\end{lstlisting}

If the \texttt{sm\_expected\_hash.h} is not present in the \texttt{include} folder, can be generated for the first time following the instruction in the \texttt{README.md} file of the SM repository available at the following url: \url{https://github.com/keystone-enclave/sm.git}. 

\subsection{Tools for provisioning}

The file \texttt{provision} can be used to generate a new \texttt{test\_client\_key.h}, which will contain the root key pair, the client key pair and the signature of the client public key made by the system administrator with its secrey key. 

It could be modified to meet your purposes (for example to generate more client key pairs and signatures or to use a fixed root key).
The program uses \texttt{libsodium} to generate an Ed25519 key pair for the system administrator. Then will generate another Ed25519 key pair for the client and the public part will be signed with the system administrator key pair. The server will use the system administrator public key to authenticate that the client is part of the system administrator network.
 
Once \texttt{libsodium}  has been installed, compile \texttt{provision.c}, run it and redirect the output to a file with the name \texttt{test\_client\_key.h}. 
The file \texttt{test\_client\_key.h} is already present in the \texttt{include} folder, but it may be necessary to generate a new one and it can be done with the following commands. \\

\begin{lstlisting}[style=terminal,frame=single]
$ git clone https://github.com/jedisct1/libsodium.git
$ cd libsodium
$ git checkout 4917510626c55c1f199ef7383ae164cf96044aea
$ ./configure
$ make && make check
$ sudo make install
$ sudo ldconfig

$ cd keystone-demo/provisioning
$ gcc -o provision provision.c -lsodium

$ ./provision > ../include/test_client_key.h
\end{lstlisting}