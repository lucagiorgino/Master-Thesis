%!TEX encoding = IsoLatin
%!TEX main = ../../main.tex

The concept of digital identity has been evolving in the last decades. Digital identity is the expression and storage of one's identity in digital form, which is a set of claims, i.e. assertions of some truths, made about a subject, which can be a person, a thing, a device, etc. Despite this, only in recent years privacy, control and ownership over digital identity and personal data are being increasingly recognized as relevant factors by individuals \cite{TheLawsOfIdentity}. 
Since the advent of the Internet, digital identity models have gone through several stages, from centralized identity to federated identity, becoming more user-centric over time reaching the definition of the Self-Sovereign Identity (SSI) paradigm \cite{ThePathToSSI}.
In the Internet's early days, digital identity issuers and authenticators were designed as centralized authorities. Unfortunately, giving centralized authorities control over a user's digital identity has several drawbacks. For example, a single authority can deny a user's identity or even confirm a false one, so the user has no control over his own identity. SSI is the next development of digital identity where the main idea is that the user must be at the centre of identity management \cite{ThePathToSSI}.


The Internet of Things increasingly involves the collection, processing and transmission of a wide variety of data to services and other devices \cite{wilson2018digital}. Reasonably obvious privacy risks arise from IoT-connected devices when they exchange identifiable information, as this can reveal the activities and behaviours of users' devices and subtle risks arise when a considerable amount of data is available for analysis and linkage to additional data sets, since user identification or re-identification may happen as a result \cite{wilson2018digital}. Digital identity, such as SSI, could be a solution for building a digital realm where heterogeneous objects and people interact securely. This will allow IoT devices to be unique and distinguishable from one another.


In this document, Self-Sovereign Identity is introduced, as well as Keystone enclave, an open-source framework for building customizable trusted execution environments based on RISC-V hardware. Then is presented the solution, which has been designed to support a constrained device to create and manage its own digital identity. 

% contesto
% problema
% soluzione

% The topics are organized as follows:
% \begin{itemize}
%     \item Chapter \ref{chap:1} 
%     \item Chapter \ref{chap:2} 
%     \item Chapter \ref{chap:3}
%     \item Chapter \ref{chap:4}
%     \item Chapter \ref{chap:5}
% \end{itemize}
