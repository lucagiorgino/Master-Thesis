%!TEX encoding = IsoLatin
%!TEX main = ../../main.tex
\section{Self-Sovereign-Identity }
Self-Sovereign Identity (SSI) \cite{tobin2016inevitable} is a new model for digital identity. In the SSI ecosystem, a user can fully control his own identity and use it between any service. SSI is different from today's digital identities: it is anchored to distributed ledgers so is not controlled by any centralized services.
One SSI innovation is the design and development of a common set of specifications: 
Decentralized Identifiers (DIDs) \cite{didW3C} and Verifiable Credentials (VCs) \cite{vcW3C}, by doing this a user identity can be anchored to different distributed ledgers but it will be defined in the same standard way.

\subsection{Decentralized Identifiers}
DIDs 
\begin{figure}
    \includesvg[inkscapelatex=false, scale=0.55]{./chapters/images/parts-of-a-did.svg}
    \caption{A simple example of a DID}
\end{figure}
\subsection{Verifiable Credentials}
VCs is pre

\section{Keystone Enclave}
Device manufacturers are now taking security concerns more seriously than they previously did as a result of the rise in popularity of networked devices in recent years.
To adequately address these challenges, a specification has been developed that defines a way to ensure the integrity and confidentiality of sensitive data in the device that implements the specification.
\cite{IntroTEE}
\subsection{Trusted Execution Environment}
A Trusted Execution Environment (TEE) is a safe area within a CPU. It runs in an isolated environment and in parallel with the operating system.
It ensures that the confidentiality and integrity of the code and data loaded in the TEE are preserved. 
Trusted applications running on TEE have access to the full capabilities of a device's main processor and memory, while hardware isolation shields these components from user-installed apps running in the main operating system. The various included trusted applications are protected from one another by software and cryptographic isolations within the TEE.
\cite{IntroTEE}
%aggiungere definizione di enclave, tcb?
The two most common TEE implementations at the moment are ARM TrustZone and Intel SGX. All these TEEs make design decisions based on either the target applications or threat models and these choices are fixed since they are strictly hardware related. They were not designed to have flexibility or extensibility for enclave developers.  If the hardware changes or has a new feature, the enclave developer has to redesign the TEE.
All TEE platforms aim to reduce the enclave's TCB, yet they have managed to achieve different degrees of success. Additionally, closed-source hardware and microcode implementations make it impossible for a third party to evaluate the security of TEEs.
\textbf{Customizable TEE} is the solution to these problems. It has been designed to be flexible, configurable and to have a small TCB. It has been developed with clear abstractions and a modular programming model which simplifies for others to extend and add features to the TEE. A customizable TEE is Keystone.
\cite{lee2020keystone} 
\subsection{Keystone overview}
Keystone \cite{lee2020keystone} is an open-source framework for creating RISC-V hardware-based Trusted Execution Environments (TEEs) that are adaptable for use on a variety of platforms. Keystone offers security primitives that can be joined together via the software framework rather than creating a single instance of TEE hardware. The TEE can be modified by the creator of the enclave and the platform provider to suit their threat models or platform configurations. The Keystone project offers a general and formally proven interface for a variety of devices to create an open standard for TEEs. Every piece of hardware could, in our opinion, have a secure TEE at practically no extra cost.
\subsubsection{Security Monitor}
\subsubsection{Runtime}