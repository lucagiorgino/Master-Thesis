%!TEX encoding = IsoLatin
%!TEX main = ../../main.tex

\section{Developer manual}
\subsection{Constrained and edge device comparison}
\subsubsection{Test MbedTLS library}
{\color{red} ToDo: spiegare come il progetto e' organizzato e file/funzioni rilevanti}
\subsubsection{Test BBS\texttt{+} signatures scheme}
For this subproject the most relevant files are:
\begin{itemize}
  \item \texttt{Cargo.toml} where dependencies are added, such as  
  \texttt{bbs} and \texttt{rand}. 
  \item \texttt{main.rs} where tests for BBS\texttt{+} key generation, signature and verification are defined. 
  \begin{itemize} 
    \item \texttt{fn key\_gen\_test(iterations: usize)}
    \item \texttt{fn simple\_sign\_ver\_test(iterations: usize)}
    % \item \texttt{fn pok\_sign\_ver\_test(iterations: usize)}
  \end{itemize}
\end{itemize}
And the code is very simple, it launches the test for key generation and the test for signature and verification, operations are executed \texttt{iterations} times and the average is displayed on the output video. 
\begin{lstlisting}[frame=single]
fn main() {
  let iterations: usize = 100;
  key_gen_test(iterations);
  simple_sign_ver_test(iterations);
}
\end{lstlisting}

\subsection{Proof of concept}
\subsection*{Guide to Keystone Components}
The Keystone repository consists of several sub-components such as gitmodules or directories. This is a brief overview of them.
\begin{lstlisting}[frame=single]
+ keystone/
|-- patches/
|  # required patches for submodules
|-- bootrom/
|  # Keystone bootROM for QEMU virt board, including trusted boot chain.
|-- buildroot/
|  # Linux buildroot. Builds a minimal working Linux image for our test platforms.
|-- docs/
|  # Contains read-the-docs formatted and hosted documentation, such as this article.
|-- riscv-gnu-toolchain/
|  # Unmodified toolchain for building riscv targets. Required to build all other components.
|-- linux-keystone-driver/
|  # A loadable kernel module for Keystone enclave.
|-- linux/
|  # Linux kernel
|-- sm/
|  # OpenSBI firmware + Keystone security monitor
|-- qemu/
|  # QEMU
+-- sdk/
    # Tools, libraries, and example apps for building enclaves on Keystone        
\end{lstlisting}

\subsection*{Guide to Keystone Demo Proof of concept}
The designed solution has been developed starting from the officially \texttt{kestone-demo} repository available at this link: \url{https://github.com/keystone-enclave/keystone-demo}. The demo uses test keys and is not safe for production use.
\begin{lstlisting}[frame=single]
  + keystone-demo-poc/
  |-- docs/
  |  # Contains read-the-docs formatted and hosted documentation, such as this article.
  |-- include/
  |  # Contains shared files between eapp and client
  |-- scripts/
  |  # Contains a script for performing the attestation of sm and eapp
  |-- server_eapp/
  |  # small enclave server that is capable of remote attestation, secure channel creation, and performing a simple word-counting computation securely
  |-- sodium_patches/
  |  # Contains patch for libsodium that will run in the server eapp
  +-- trusted_client/
     # simple remote client that connects to the host, validates the enclave report, constructs a secure channel, and then can send messages to the host for computation.       
  \end{lstlisting}

\subsection*{Relevant files of the demo}
Below are explained relevant files that change from the official \texttt{kestone-demo} repository.

\subsubsection{include/eh\_shared.h}
This file is shared between the enclave and the untrusted host. It defines the following data structure for exchanging sealed (encrypted) data between the enclave and the untrusted host. 

\begin{lstlisting}[language=C,frame=single]
#define FILE_CLIENT_KEYS_SIGNATURE 1
#define FILE_CLIENT_VC_SIGNATURE 2

typedef struct stored_data_t{
  unsigned short file_type;
  unsigned char client_pk[crypto_kx_PUBLICKEYBYTES];
  size_t c_len; // content len 
  unsigned char content[]; // Flexible member
} stored_data_t;    
\end{lstlisting}

\subsubsection{include/messages.h}
This file is shared between the client and the enclave and it defines the messages (request amd response) they exchange. 
\begin{lstlisting}[language=C,frame=single]
#define SERVICE_GEN_KEYS 1
#define SERVICE_STORE_VC 2
#define SERVICE_GET_VP 3
#define MSG_EXIT 4

typedef struct request_message_t {
  unsigned short request_type;
  unsigned char secret[SECRET_LEN];
  size_t len;
  unsigned char payload[]; // Flexible member
} request_message_t;

typedef struct response_message_t {
  unsigned short response_type;
  size_t len;
  unsigned char payload[]; // Flexible member
} response_message_t;
\end{lstlisting}

\subsubsection{include/session\_context.c and include/session\_context.h}
The session context is the structure that the client provides to the enclave to prove that is built by a trusted manufacturer. With the function \texttt{session\_context\_from\_buffer} the session context is extracted from the received buffer. 
\begin{lstlisting}[language=C,frame=single]
struct session_context_t {
  unsigned char  dh_public_key[PUBLIC_KEY_SIZE];
  unsigned char  challenge[CHALLENGE_SIZE];
  unsigned char  data_signature[SIGNATURE_SIZE];
  
  unsigned char  client_public_key[PUBLIC_KEY_SIZE];
  unsigned char  root_signature_of_client_pk[SIGNATURE_SIZE];
};

void session_context_from_buffer(struct session_context_t* session_context, unsigned char* buffer);
\end{lstlisting}

\noindent
\texttt{int session\_context\_verify( ... )}\\
\textit{Input}:
\begin{itemize}[noitemsep,nolistsep]
  \item \texttt{struct session\_context\_t session\_context}, the session context to verify
  \item \texttt{unsigned char* challange}, the challenge that the server has sent to the client
  \item \texttt{const unsigned char* root\_public\_key}, the public key of the manufacturer
\end{itemize}
\textit{Output}: an \texttt{int} value, 1 if it is a valid session context, 0 if not.

\subsubsection{server\_eapp/service.c and server\_eapp/service.h}
The file \texttt{service.h} just exposes the function that processes the request received by the client.  
\begin{lstlisting}[language=C,frame=single]
response_message_t* process_request(request_message_t *request, size_t *pt_finalsize) {

  setup_sealing_material(request->secret);

  switch (request->request_type) {
  case SERVICE_GEN_KEYS:
    return generate_public_keys(pt_finalsize, EdDSA);
    break;
  case SERVICE_STORE_VC:
    return store_verifiable_credential(pt_finalsize, request->payload, request->len);
    break;
  case SERVICE_GET_VP:
    return get_verifiable_presentation(pt_finalsize, request->payload, request->len, EdDSA);
    break;
  }
  return NULL;
}
\end{lstlisting}
{\color{red} ToDo: esplicitare cosa indicano i valori in input e qual e' l'output}\\
\noindent
\texttt{void store\_data ( ... )}\\
It saves the sealed data in untrusted non-volatile memory. \\
\textit{Input}:
\begin{itemize}[noitemsep,nolistsep]
  \item \texttt{unsigned char* buffer}, data to save
  \item \texttt{size\_t len}, lenght of the data to save
  \item \texttt{unsigned short file\_type}, type of the data to save (possible values: \texttt{FILE\_CLI\-ENT\_KEYS\_SIGNATURE} or \texttt{FILE\_CLIENT\_VC\_SIGNATURE}\)
\end{itemize}
\textit{Output}: none

\noindent
\texttt{unsigned char* seal\_data\_and\_sign ( ... )}\\
It encrypts and signs the data to save in untrusted non-volatile memory. \\
\textit{Input}:
\begin{itemize}[noitemsep,nolistsep]
  \item \texttt{unsigned char* data}, data to encrypt and sign
  \item \texttt{size\_t data\_len}, length of the data to encrypt and sign
  \item \texttt{size\_t* sign\_len}, pointer to store the actual length of the signed message
\end{itemize}
\textit{Output}: the signed message, which includes the signature plus an unaltered copy of the message. 


\noindent
\texttt{response\_message\_t* build\_response ( ... )}\\
It builds the response to sand back to the client. \\
\textit{Input}:
\begin{itemize}[noitemsep,nolistsep]
\item \texttt{size\_t* pt\_finalsize}, pointer to store the final length of the response
\item \texttt{unsigned short response\_type}, type of the response (possible values: \texttt{SERVICE\_\-GEN\_KEYS}, \texttt{SERVICE\_STORE\_VC}, \texttt{SERVICE\_GET\_VP}, \texttt{MSG\_EXIT})
\item \texttt{unsigned char *buffer}, response to send back to the client
\item \texttt{size\_t len}, length of the response
\end{itemize}
\textit{Output}: the built response 


\noindent
\texttt{response\_message\_t* generate\_public\_keys ( ... )}\\
\textit{Input}:
\begin{itemize}[noitemsep,nolistsep]
\item \texttt{size\_t* pt\_finalsize}
\item \texttt{int key\_type}
\end{itemize}
\textit{Output}: 

\noindent
\texttt{response\_message\_t* store\_verifiable\_credential ( ... )}\\
\textit{Input}:
\begin{itemize}[noitemsep,nolistsep]
  \item \texttt{size\_t* pt\_finalsize}
  \item \texttt{unsigned char* vc}
  \item \texttt{size\_t vc\_len}
\end{itemize}
\textit{Output}: 

\noindent
\texttt{response\_message\_t* get\_verifiable\_presentation ( ... )}\\
\textit{Input}:
\begin{itemize}[noitemsep,nolistsep]
  \item \texttt{size\_t* pt\_finalsize}
  \item \texttt{unsigned char* nonce}
  \item \texttt{size\_t nonce\_len}
  \item \texttt{int key\_type}
\end{itemize}
\textit{Output}: 

\subsubsection{server\_eapp/server\_eapp.c}
In this file the important functions to mention are:
\begin{itemize}
  %\item \texttt{void validate\_session\_context(void* buffer, unsigned char* challange)}
  \item \texttt{void attest\_and\_establish\_channel()}
  \item \texttt{void handle\_requests()}
\end{itemize}
{\color{red} ToDo: spiegare queste funzioni}\\

\subsubsection{trusted\_client/trusted\_client.c and trusted\_client/trusted\_client.h}
In this file the important functions to mention are:

\noindent
\texttt{void gen\_session\_context( ... )}\\
\textit{Input}:
\begin{itemize}[noitemsep,nolistsep]
  \item \texttt{byte* buffer}
\end{itemize}
\textit{Output}: 


\noindent
\texttt{request\_message\_t* generate\_request\_message( ... )}\\
\textit{Input}:
\begin{itemize}[noitemsep,nolistsep]
  \item \texttt{char* buffer}
  \item \texttt{size\_t buffer\_len}
  \item \texttt{size\_t* finalsize}
  \item \texttt{unsigned char* secret}
  \item \texttt{unsigned short request\_type}
\end{itemize}
\textit{Output}: 


\noindent
\texttt{request\_message\_t* generate\_exit\_message( ... )}\\
\textit{Input}:
\begin{itemize}[noitemsep,nolistsep]
  \item \texttt{size\_t* finalsize}
\end{itemize}
\textit{Output}: 


%%%%%%%%%%%%%%%%%%%%%%%%%%%%%%%%%%%%%%%%%%%%%%%%%%%%%%%%%%%%%%%%%%%%%%%%%%%%%%%%%%%%%%%%%%%%%%%%%%%%%%%%%%%%%%%%%%%%%%%%%%%%%%%%%%%%%%%%%%%%%%%%%%%%%%%%%%%%%%
% \begin{lstlisting}[language=C,frame=single]
   
% \end{lstlisting}

% \noindent
% \texttt{}\\
% \textit{Input}:\setlist{nolistsep}
% \begin{itemize}[noitemsep]
%   \item 
% \end{itemize}
% \textit{Output}: 
