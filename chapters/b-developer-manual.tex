%!TEX encoding = IsoLatin
%!TEX main = ../../main.tex

\section{Constrained and edge device comparison}
\subsection{Test MbedTLS library}
\texttt{mytest.c} is the most relevant file for this subproject. It is equal for both non-constrained devices and the constrained device (STM32L4+ board \cite{stm32-board-product}). The only difference is the function used for measuring the elapsed time of a cryptographic primitive. The function \texttt{clock\_gettime()}, of the standard library \texttt{time.h}, is used on the non-constrained device, while on the STM32L4+ board is used the \texttt{HAL\_GetTick()}. Figure \ref{pseudo-algo} describes in a pseudo-language how to measure the elapsed time of a cryptographic primitive. 
Then, the available tests are described. \\
\begin{figure}[H]
\begin{lstlisting}[language=C,frame=single]
  start = clock_gettime( ) or HAL_GetTick()
  // mbedtls cryptographic primitive
  end = clock_gettime( ) or  HAL_GetTick()
  print ( end - start ) // in ms
\end{lstlisting}
\caption{Pseudo algorithm for measuring the elapsed time. \label{pseudo-algo}}
\end{figure}

\noindent
\texttt{\bfseries int test\_RSA\_keygen(\dots)}\\
It generates \texttt{N} RSA key pair and prints the elapsed for each generation in a CSV format. \\
\textit{Input}:
\begin{itemize}[noitemsep,nolistsep]
  \item \texttt{int key\_size}, size of the generated key (possible values: \texttt{2048}, \texttt{3072} or \texttt{4096})
  \item \texttt{mbedtls\_ctr\_drbg\_context *ctr\_drbg}, reference to \texttt{CTR\_DRBG} context structure for random number generation
  \item \texttt{int iterations}, number of times that the test will be executed
\end{itemize}
\textit{Output}: Returns \texttt{1} if successful, or an \texttt{MBEDTLS\_ERR\_XXX\_XXX} error code \\


\noindent
\texttt{\bfseries int test\_RSA(\dots)}\\
It generates an RSA key pair, computes the RSA signature of a random hashed message (of 1024 bytes) and verifies the generated signature. Then, it prints the elapsed for each signature and verification in a CSV format. (Everything is repeated \texttt{N} times). \\
\textit{Input}:
\begin{itemize}[noitemsep,nolistsep]
  \item \texttt{int key\_size}, size of the generated key (possible values: \texttt{2048}, \texttt{3072} or \texttt{4096})
  \item \texttt{int sha\_alg},
  \item \texttt{mbedtls\_ctr\_drbg\_context *ctr\_drbg}, reference to \texttt{CTR\_DRBG} context structure for random number generation
  \item \texttt{int iterations}, number of times that the test will be executed
\end{itemize}
\textit{Output}:  Returns \texttt{1} if successful, or an \texttt{MBEDTLS\_ERR\_XXX\_XXX} error code \\


\noindent
\texttt{\bfseries int test\_ECDSA\_keygen(\dots)}\\
It generates \texttt{N} EDSA key pair and prints the elapsed for each generation in a CSV format. \\
\textit{Input}:
\begin{itemize}[noitemsep,nolistsep]
  \item \texttt{int ecparams}, the identifier of domain parameters (curve, subgroup and generator),(possible values: \texttt{MBEDTLS\_ECP\_DP\_SECP256R1}, \texttt{MBEDTLS\_ECP\_DP\_SECP384R1} or \texttt{MBEDTLS\_ECP\_DP\_SECP521R1})
  \item \texttt{int sha\_alg},
  \item \texttt{mbedtls\_ctr\_drbg\_context *ctr\_drbg}, reference to \texttt{CTR\_DRBG} context structure for random number generation
  \item \texttt{int iterations}, number of times that the test will be executed
\end{itemize}
\textit{Output}:  Returns \texttt{1} if successful, or an \texttt{MBEDTLS\_ERR\_XXX\_XXX} error code \\


\noindent
\texttt{\bfseries int test\_ECDSA(\dots)}\\
It generates an ECDSA key pair, computes the ECDSA signature of a random hashed message (of 1024 bytes) and verifies the generated signature. Then, it prints the elapsed for each signature and verification in a CSV format. (Everything is repeated \texttt{N} times). \\
\textit{Input}:
\begin{itemize}[noitemsep,nolistsep]
  \item \texttt{int ecparams}, the identifier of domain parameters (curve, subgroup and generator),(possible values: \texttt{MBEDTLS\_ECP\_DP\_SECP256R1}, \texttt{MBEDTLS\_ECP\_DP\_SECP384R1} or \texttt{MBEDTLS\_ECP\_DP\_SECP521R1})
  \item \texttt{int sha\_alg}, digest algorithm (possible values: \texttt{MBEDTLS\_MD\_SHA256}, \texttt{MBEDTLS\_MD\_SHA384} or \texttt{MBEDTLS\_MD\_SHA512})
  \item \texttt{mbedtls\_ctr\_drbg\_context *ctr\_drbg}, reference to \texttt{CTR\_DRBG} context structure for random number generation
  \item \texttt{int iterations}, number of times that the test will be executed
\end{itemize}
\textit{Output}:  Returns \texttt{1} if successful, or an \texttt{MBEDTLS\_ERR\_XXX\_XXX} error code
\newpage
% ===========================================================================
% SUBSECTION
% ===========================================================================

\subsection{Test BBS\texttt{+} signatures scheme}
For this subproject, the most relevant files are:
\begin{itemize}
  \item \texttt{Cargo.toml} where dependencies are added, such as  
  \texttt{bbs} and \texttt{rand}. 
  \item \texttt{main.rs} where tests for BBS\texttt{+} key generation, signature and verification are defined. 
  \begin{itemize} 
    \item \texttt{fn key\_gen\_test(iterations: usize)}
    \item \texttt{fn simple\_sign\_ver\_test(iterations: usize)}
    % \item \texttt{fn pok\_sign\_ver\_test(iterations: usize)}
  \end{itemize}
\end{itemize}

Figure \ref{bbs-pseudo-algo} describes in Rust language how to measure the elapsed time of a cryptographic primitive. \\
\begin{figure}[H]
\begin{lstlisting}[frame=single]
fn <primitive>_test(iterations: usize) {
  let mut sum: u128 = 0;
  for i in 0..iterations {
      let start = Instant::now();
      // bbs+ cryptographic primitive
      let duration = start.elapsed().as_millis();
      sum += duration;
  };
  println!("AVG Time elapsed is: {:?}", sum/iterations as u128);
}
    
\end{lstlisting}
\caption{Pseudo algorithm for measuring the elapsed time.}
\label{bbs-pseudo-algo}
\end{figure}
The program does the following, it launches the test for key generation and the test for signature and verification, operations are executed \texttt{iterations} times and the average is displayed on the output video.  \\
\begin{lstlisting}[frame=single]
fn main() {
  let iterations: usize = 100;
  key_gen_test(iterations);
  simple_sign_ver_test(iterations);
}
\end{lstlisting}


% ===========================================================================
% SUBSECTION
% ===========================================================================
\newpage
\section{Proof of concept}
\subsection{Guide to Keystone Components}
The Keystone repository consists of several sub-components such as gitmodules or directories. This is a brief overview of them. \\
\begin{lstlisting}[frame=single]
+ keystone/
|-- patches/
|  # required patches for submodules
|-- bootrom/
|  # Keystone bootROM for QEMU virt board, including trusted boot chain.
|-- buildroot/
|  # Linux buildroot. Builds a minimal working Linux image for our test platforms.
|-- docs/
|  # Contains read-the-docs formatted and hosted documentation, such as this article.
|-- riscv-gnu-toolchain/
|  # Unmodified toolchain for building riscv targets. Required to build all other components.
|-- linux-keystone-driver/
|  # A loadable kernel module for Keystone enclave.
|-- linux/
|  # Linux kernel
|-- sm/
|  # OpenSBI firmware + Keystone security monitor
|-- qemu/
|  # QEMU
+-- sdk/
  # Tools, libraries, and example apps for building enclaves on Keystone        
\end{lstlisting}
\newpage
\subsection{Guide to Keystone Demo Proof of concept}
The designed solution has been developed starting from the officially \texttt{kestone-demo} repository available at this link: \url{https://github.com/keystone-enclave/keystone-demo}, commit \texttt{8c6c0565e44f3d0e00bc3e4a6e77fc84c9e6d343}. The demo uses test keys and is not safe for production use. \\
\begin{lstlisting}[frame=single]
+ keystone-demo-poc/
|-- docs/
|  # Contains read-the-docs formatted and hosted documentation, such as this article.
|-- include/
|  # Contains shared files between eapp and client
|-- provisioning/
|  # C program to generate new client and root key pairs
|-- scripts/
|  # Contains a script for performing the attestation of sm and eapp
|-- server_eapp/
|  # small enclave server that is capable of remote attestation, secure channel creation, and performing a simple word-counting computation securely
|-- sodium_patches/
|  # Contains patch for libsodium that will run in the server eapp
+-- trusted_client/
  # simple remote client that connects to the host, validates the enclave report, constructs a secure channel, and then can send messages to the host for computation.       
\end{lstlisting}
\newpage
\subsection{Relevant files of the demo}
Below are explained relevant files that change from the official \texttt{kestone-demo} repository.
Functions, types and variables not mentioned here can be deepened in the official \texttt{kestone-demo} repository and Keystone documentation.
\subsubsection{include/eh\_shared.h}
This file is shared between the enclave and the untrusted host. It defines the following data structure for exchanging sealed (encrypted) data between the enclave and the untrusted host. \\

\begin{lstlisting}[language=C,frame=single]
typedef struct stored_data_t{
  unsigned short file_type;
  unsigned char client_pk[crypto_kx_PUBLICKEYBYTES];
  size_t c_len; // content len 
  unsigned char content[]; // Flexible member
} stored_data_t;    
\end{lstlisting}

\subsubsection{include/messages.h}
This file is shared between the client and the enclave and it defines the messages (request and response) they exchange. \\
\begin{lstlisting}[language=C,frame=single]
typedef struct request_message_t {
  unsigned short request_type;
  unsigned char secret[SECRET_LEN];
  size_t len;
  unsigned char payload[]; // Flexible member
} request_message_t;

typedef struct response_message_t {
  unsigned short response_type;
  size_t len;
  unsigned char payload[]; // Flexible member
} response_message_t;
\end{lstlisting}
\newpage
\subsubsection{include/session\_context.c and include/session\_context.h}
The session context is the structure that the client provides to the enclave to prove that is built by a trusted manufacturer. With the function \texttt{session\_context\_from\_buffer} the session context is extracted from the received buffer. \\
\begin{lstlisting}[language=C,frame=single]
struct session_context_t {
  unsigned char  dh_public_key[PUBLIC_KEY_SIZE];
  unsigned char  challenge[CHALLENGE_SIZE];
  unsigned char  data_signature[SIGNATURE_SIZE];
  
  unsigned char  client_public_key[PUBLIC_KEY_SIZE];
  unsigned char  root_signature_of_client_pk[SIGNATURE_SIZE];
};

void session_context_from_buffer(struct session_context_t* session_context, unsigned char* buffer);
\end{lstlisting}

\noindent
\texttt{\bfseries int session\_context\_verify(\dots)}\\
\textit{Input}:
\begin{itemize}[noitemsep,nolistsep]
  \item \texttt{struct session\_context\_t session\_context}, the session context to verify
  \item \texttt{unsigned char* challange}, the challenge that the server has sent to the client
  \item \texttt{const unsigned char* root\_public\_key}, the public key of the manufacturer
\end{itemize}
\textit{Output}: an \texttt{int} value, 1 if it is a valid session context, 0 if not.


% ===========================================================================
% SUBSUBSECTION
% ===========================================================================
\newpage
\subsubsection{server\_eapp/service.c and server\_eapp/service.h}
\label{service.c}
The file \texttt{service.h} just exposes the function that processes the request received by the client. \\   
\begin{lstlisting}[language=C,frame=single]
response_message_t* process_request(request_message_t *request, size_t *pt_finalsize) {

  setup_sealing_material(request->secret);

  switch (request->request_type) {
    case SERVICE_GEN_KEYS:
      return generate_public_keys(pt_finalsize, EdDSA);
      break;
    case SERVICE_STORE_VC:
      return store_verifiable_credential(pt_finalsize, request->payload, request->len);
      break;
    case SERVICE_GET_VP:
      return get_verifiable_presentation(pt_finalsize, request->payload, request->len, EdDSA);
      break;
    default:  
      ocall_print_buffer("Invalid request type!\n");
  }
  return NULL;
}
\end{lstlisting}
% \leavevmode\newline

\noindent
\texttt{\bfseries int store\_data (\dots)}\\
It saves the sealed data in untrusted non-volatile memory. \\
\textit{Input}:
\begin{itemize}[noitemsep,nolistsep]
  \item \texttt{unsigned char* buffer}, data to save
  \item \texttt{size\_t len}, length of the data to save
  \item \texttt{unsigned short file\_type}, type of the data to save (possible values: \texttt{FILE\_CLI\-ENT\_KEYS\_SIGNATURE} or \texttt{FILE\_CLIENT\_VC\_SIGNATURE})
\end{itemize}
\textit{Output}: an \texttt{int} value, 1 if the operation is successful, 0 if not. \\

\noindent
\texttt{\bfseries unsigned char* seal\_data\_and\_sign (\dots)}\\
It encrypts and signs the data to save in untrusted non-volatile memory. \\
\textit{Input}:
\begin{itemize}[noitemsep,nolistsep]
  \item \texttt{unsigned char* data}, data to encrypt and sign
  \item \texttt{size\_t data\_len}, length of the data to encrypt and sign
  \item \texttt{size\_t* sign\_len}, a pointer to store the actual length of the signed message
\end{itemize}
\textit{Output}: the signed message, which includes the signature plus an unaltered copy of the message. \\


\noindent
\texttt{\bfseries response\_message\_t* build\_response (\dots)}\\
It builds the response to sand back to the client. \\
\textit{Input}:
\begin{itemize}[noitemsep,nolistsep]
\item \texttt{size\_t* pt\_finalsize}, pointer to store the final length of the response
\item \texttt{unsigned short response\_type}, type of the response (possible values: \texttt{SERVICE\_\-GEN\_KEYS}, \texttt{SERVICE\_STORE\_VC}, \texttt{SERVICE\_GET\_VP}, \texttt{MSG\_EXIT})
\item \texttt{unsigned char *buffer}, response to send back to the client
\item \texttt{size\_t len}, length of the response
\end{itemize}
\textit{Output}: the built response \\


\noindent
\texttt{\bfseries response\_message\_t* generate\_public\_keys (\dots)}\\
It handles the request of the client to generate two key pairs. It saves the key pairs sealed in the untrusted non-volatile memory and sends back to the client only the public part. \\ 
\textit{Input}:
\begin{itemize}[noitemsep,nolistsep]
\item \texttt{size\_t* pt\_finalsize}, pointer to store the final length of the response
\item \texttt{int key\_type}, the key type that the client chooses to generate 
% (for demo purposes only \texttt{EdDSA} can be generated, in future \texttt{BBS+} signature scheme will be implemented)
\end{itemize}
\textit{Output}: the built response \\

\noindent
\texttt{\bfseries response\_message\_t* store\_verifiable\_credential (\dots)}\\
It handles the request of the client to store a verifiable credential that the client obtained from an issuer. It saves the verifiable credential sealed in the untrusted non-volatile memory and sends back to the client a \texttt{0} if everything goes right. \\
\textit{Input}:
\begin{itemize}[noitemsep,nolistsep]
  \item \texttt{size\_t* pt\_finalsize}, pointer to store the final length of the response
  \item \texttt{unsigned char* vc}, verifiable credential to be stored
  \item \texttt{size\_t vc\_len}, length of the verifiable credential to be stored
\end{itemize}
\textit{Output}: the built response \\

\noindent
\texttt{\bfseries response\_message\_t* get\_verifiable\_presentation (\dots)}\\
It handles the request of the client to generate a verifiable presentation to use when interacting with a verifier. It retrieves from the untrusted non-volatile memory the previously stored verifiable credential and key pairs and sends back to the client the sign of the verifiable credential with one using an assertion key of the type that the client chose. \\
\textit{Input}:
\begin{itemize}[noitemsep,nolistsep]
  \item \texttt{size\_t* pt\_finalsize}, pointer to store the final length of the response
  \item \texttt{unsigned char* nonce}, \textit{(optional)} nonce that the verifier asks the client to insert in the verifiable presentation  
  \item \texttt{size\_t nonce\_len}, length of nonce
  \item \texttt{int key\_type}, the key type that the client previously chose to generate 
  % (for demo purposes only \texttt{EdDSA} can be used, in future \texttt{BBS+} signature scheme will be implemented)
\end{itemize}
\textit{Output}: the built response 

% ===========================================================================
% SUBSUBSECTION
% ===========================================================================
\newpage
\subsubsection{server\_eapp/server\_eapp.c}
In this file, the important functions to mention are:

\noindent
\texttt{\bfseries void attest\_and\_establish\_channel()}\\
In this function, as explained in Chap. \ref{chap:SSIaaS}, the enclave sends its report to the client and waits for the session context of the client. Once received, it validates the session context and established the channel with the client. If some error occurs, the program terminates. \\

\begin{lstlisting}[language=C,frame=single]
void attest_and_establish_channel() {
  // ...  
  randombytes_buf(challenge, CHALLENGE_SIZE);
  // ...  
  ocall_send_report(buffer, MAX_REPORT_SIZE);

  unsigned char session_ctx_buffer[SESSION_CTX_SIZE];
  ocall_wait_for_client_session_ctx(session_ctx_buffer, SESSION_CTX_SIZE);
  // signatures validity check
  validate_session_context(session_ctx_buffer, challenge); 

  channel_establish(); // Ask libsodium to generate session keys based on the received pk
} 
\end{lstlisting}
% \leavevmode\newline
\newpage
\noindent
\texttt{\bfseries void handle\_requests()}\\
In this function, as explained in Chap. \ref{chap:SSIaaS} and in Sect. \ref{service.c}, the enclave will wait for client requests. Whenever a request arrives, the enclave copies the request from the shared region, processes the request (interacting with the enclave host for data sealing), and sends the response back to the client. If some error occurs, the program terminates. \\

\begin{lstlisting}[language=C,frame=single]
void handle_requests() {
  struct edge_data msg;
  while(1){
    ocall_wait_for_request(&msg);
    // ...
    copy_from_shared(request, msg.offset, msg.size);
    // ...
    switch (request->request_type) {
      case MSG_EXIT:
        ocall_print_buffer("Received exit, exiting\n");
        EAPP_RETURN(0);
      break;
      default:
        response = process_request(request, &r_size);
      break;
    }
    if (response == NULL) {
      ocall_print_buffer("Response handling error\n");
      EAPP_RETURN(0);
    }
    // ...
    channel_send((unsigned char*) response, r_size, boxed_buffer);
    ocall_send_reply(boxed_buffer, boxed_size);
    // ... 
  }
}
\end{lstlisting}
    

%\item \texttt{void validate\_session\_context(void* buffer, unsigned char* challange)}



% ===========================================================================
% SUBSUBSECTION
% ===========================================================================
\newpage
\subsubsection{trusted\_client/trusted\_client.c and trusted\_client/trusted\_client.h}
In this file, the important functions to mention are:

\noindent
\texttt{\bfseries int gen\_session\_context(\dots)}\\
It generates the client session context that will contain a data part and the signature of the client's public key made by the system administrator at the provisioning phase. The data part is composed of a key for the Diffie Hellam key exchange protocol and the challenge sent by the server, all the data part is signed by the client's public key. \\
\textit{Input}:
\begin{itemize}[noitemsep,nolistsep]
  \item \texttt{byte* buffer}, a pointer to the buffer that will contain the session context
\end{itemize}
\textit{Output}: an \texttt{int} value, 1 if the operation is successful, 0 if not. \\


\noindent
\texttt{\bfseries request\_message\_t* generate\_request\_message(\dots)}\\
It builds the request message to request a service to the server. \\
\textit{Input}:
\begin{itemize}[noitemsep,nolistsep]
  \item \texttt{char* buffer}, a buffer of the data to send to the server (it can contain the key type or the verifiable credential)
  \item \texttt{size\_t buffer\_len}, length of the data
  \item \texttt{size\_t* finalsize}, a pointer to store the final length of the request message
  \item \texttt{unsigned char* secret}, the secret that the server will use to generate the sealing key in the key derivation function
  \item \texttt{unsigned short request\_type}, the request type that the client has chosen to send to the server (possible values: \texttt{SERVICE\_GEN\_KEYS}, \texttt{SERVICE\_STORE\_VC} or \texttt{SERVICE\_GET\_VP})
\end{itemize}
\textit{Output}: the built request message \\

\noindent
\texttt{\bfseries request\_message\_t* generate\_exit\_message(\dots)}\\
It builds the request message to close the connection with the server. \\
\textit{Input}:
\begin{itemize}[noitemsep,nolistsep]
  \item \texttt{size\_t* finalsize}, a pointer to store the final length of the request message
\end{itemize}
\textit{Output}: the built exit request message


% ===========================================================================
% SUBSECTION
% ===========================================================================

% \begin{lstlisting}[language=C,frame=single]
   
% \end{lstlisting}

% \noindent
% \texttt{}\\
% \textit{Input}:\setlist{nolistsep}
% \begin{itemize}[noitemsep]
%   \item 
% \end{itemize}
% \textit{Output}: 
